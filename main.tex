\documentclass[
 jor,
 amsmath,amssymb,preprint,
 %reprint,
%author-year,%
%author-numerical,%
]{revtex4-2}

\usepackage{subfig}
\usepackage{color}
\usepackage{graphicx}% Include figure files
\usepackage{dcolumn}% Align table columns on decimal point
\usepackage{bm}% bold math
%\usepackage[mathlines]{lineno}% Enable numbering of text and display math
%\linenumbers\relax % Commence numbering lines
\usepackage{hyperref}
\usepackage{verbatim}

\def\d{\mathrm{d}}
\def\p{\partial}

\begin{document}

\preprint{AIP/123-QED}

\title[Probe Varying Gravitational Constant via GW]{Probe Varying Gravitational Constant via GW}% Force line breaks with \\



\author{Bing Sun}
\affiliation{%
CAS Key Laboratory of Theoretical Physics, Institute of Theoretical Physics, Chinese Academy of Sciences, Beijing 100190, China%\\This line break forced% with \\
}%

\date{\today}% It is always \today, today,
             %  but any date may be explicitly specified

\begin{abstract}
In this note, we give the effect of a varying gravitational constant $G$ on the gravitational waves (GWs) during propagation, which finally gives a very simple amplitude modification $\sqrt{\frac{G^{(o)}}{G^{(s)}}}$ and a phase correction proportional to ${\partial_t^2}\ln G$ ($\nabla^2 \ln G$) and $1/\omega$ for observation position.
%
\end{abstract}

\keywords{gravitational constant, gravitational waves}

\maketitle

\section{Action method analogous to Electromagnetic Field}
The Fierz-Pauli action for the field variable $h_{ab}$ is given by~\cite{roy2020probing}
\begin{equation}
\begin{aligned}
S = & \frac{1}{64 \pi G} \int d^4 x\left(-\partial_a h_{b c} \partial^a h^{b c}+\partial_a h_b^b \partial^a h_c^c-2 \partial_a h_c^a \partial^c h_b^b+2 \partial_a h_c^a \partial_b h^{b c}\right)
\end{aligned}
\end{equation}
We set gravitational constant $G$ as a function in a form $\phi^{-1}$ so that the action becomes
\begin{equation}
\begin{aligned}
S = & \frac{1}{64 \pi} \int d^4 x~\phi\left(-\partial_a h_{b c} \partial^a h^{b c}+\partial_a h_b^b \partial^a h_c^c-2 \partial_a h_c^a \partial^c h_b^b+2 \partial_a h_c^a \partial_b h^{b c}\right) \label{origact}
\end{aligned}
\end{equation}
In the Lorentz gauge of transverse trace variable $\bar{h}_{ab} = h_{ab} - \frac{1}{2}\eta_{ab}h$, $\partial^b \bar{h}_{ab} = 0$, the action \eqref{origact} has a simple form
\begin{equation}
\begin{aligned}
S = & \frac{1}{64 \pi} \int d^4 x~\phi \left( -\partial_a\bar{h}_{bc}\partial^a\bar{h}^{bc} + \frac{1}{2}\partial_c\bar{h}\partial^c\bar{h} \right)
\end{aligned}
\end{equation}
The equation of motion (EOM) for this action becomes
\begin{align}
\square h_{ab} + \partial_ch_{ab}\partial^c\ln\phi = 0 \label{fereom}
\end{align}
where $h_{ab} = \bar{h}_{ab} - \frac{1}{2} \eta_{ab}\bar{h}$. We separate the variable as $h_{ab} = H_{ab}e^{-i\omega t}$
\begin{equation}
\begin{aligned}
\nabla^2 H_{ab} + \omega^2 H_{ab} + i\omega \partial_t\ln \phi H_{ab} + \delta^{ij} \partial_j\ln \phi \partial_iH_{ab} = 0
\end{aligned}
\end{equation}
If the GW travels along $z$-axis in a transverse-traceless (TT) gauge, $H_{ij} = H_{ij}(z)$ and the equation becomes
\begin{align}
\partial_z^2 H_{ij} + \omega^2 H_{ij} + i\omega \partial_t\ln \phi H_{ij} + \partial_z\ln \phi \partial_zH_{ij} = 0 \label{ferorg}
\end{align}
Furthermore, if $\phi = \phi(\vec{r})$, the expression eq.\eqref{ferorg} is solvable
\begin{equation}
\begin{aligned}
\partial_z^2 H_{ij} + \partial_z \ln\phi\,\partial_z H_{ij} + \omega^2 H_{ij} = 0 \,,
\end{aligned}
\end{equation}
\begin{comment}
giving rise to, in terms of eq.\eqref{sptsol} in Appendix \ref{sec:appa} with $y(z) = -\frac{1}{2}\partial_z^2g(z)$, $g(z) = \ln\phi$,
\begin{equation}
\begin{aligned}
h^{(o)}_{ab}(\omega) = & \sqrt{\frac{G_o}{G_s}} \exp\bigg[ - \frac{i z\partial_z^2 \ln\phi}{4\omega} \bigg] h_{ab}^{(s)}(\omega) = \sqrt{\frac{G_o}{G_s}} \exp\bigg[i\omega z\frac{\partial_z^2 \ln G_o}{4\omega^2} \bigg] h_{ab}^{(s)}(\omega) \,. \label{fersptsol}
\end{aligned}
\end{equation}
\textcolor{red}{The general form is ??}
\begin{equation}
h^{(o)}_{ab}(\omega) = \sqrt{\frac{G_o}{G_s}} \exp\bigg[i\frac{ \nabla^2 \ln G_o}{4\omega^2} \vec{k}\cdot\vec{r} \bigg] h_{ab}^{(s)}(\omega) ??
\end{equation}

Now we consider a special case
\begin{equation}
G = G_0(1+\alpha(1+r/\lambda) e^{-r/\lambda})
\end{equation}
or we use the symbol $z$
\begin{equation}
G = G_s(1+\alpha(1+z/\lambda) e^{-z/\lambda})
\end{equation}
where $z$ in radiation frame is equivalent to $r$ in source frame. Thus the function $y(z)$ is written as
\begin{equation}
y(z) = \frac{1}{2} \partial_z^2 \ln G = -\frac{\alpha}{2\lambda^2}e^{-z/\lambda} \bigg(1 - \frac{z}{\lambda} \bigg) \,.
\end{equation}
and substituting it into eq.\eqref{desfirorddiffeq} with $\xi = z$ we obtain
\begin{equation}
\begin{aligned}
h = & A\cos(\omega z) + B \sin(\omega z) \\
& + \frac{\alpha e^{-z/\lambda}}{2(1+4\omega^2\lambda^2)^2} \bigg[ 2\omega z(1 + \frac{2\lambda}{z} + 4\omega^2\lambda^2)(A\sin(\omega z)-B\cos(\omega z)) \\
& - (1 + \frac{z}{\lambda} -4\omega^2\lambda^2(1 - \frac{z}{\lambda}))(A\cos(\omega z)+B\sin(\omega z))\bigg] \,.
\end{aligned}
\end{equation}
The in-going wave condition $h = \mathcal{A}e^{i\omega z}$ requires $B = iA, A = \mathcal{A}$ and $h$ becomes
\begin{equation}
\begin{aligned}
h = & \mathcal{A}e^{i\omega z} + \mathcal{A}\frac{\alpha e^{-z/\lambda}}{2(1+4\omega^2\lambda^2)^2} \bigg[ -2i\omega z(1 + \frac{2\lambda}{z} + 4\omega^2\lambda^2)e^{i\omega z} - (1 + \frac{z}{\lambda} -4\omega^2\lambda^2(1 - \frac{z}{\lambda}))e^{i\omega z}\bigg] \\
= & \mathcal{A}e^{i\omega z} \bigg[1 - i\omega z\frac{\alpha e^{-z/\lambda}}{(1+4\omega^2\lambda^2)^2} \bigg(1 + \frac{2\lambda}{z} + 4\omega^2\lambda^2\bigg) - \frac{\alpha e^{-z/\lambda}}{2(1+4\omega^2\lambda^2)^2} \bigg(1 + \frac{z}{\lambda} -4\omega^2\lambda^2(1 - \frac{z}{\lambda})\bigg)  \bigg] \\
= & \mathcal{A}e^{i\omega z} \exp\bigg[- i\omega z\frac{\alpha e^{-z/\lambda}}{(1+4\omega^2\lambda^2)^2} \bigg(1 + \frac{2\lambda}{z} + 4\omega^2\lambda^2\bigg) \bigg] \exp\bigg[- \frac{\alpha e^{-z/\lambda}}{2(1+4\omega^2\lambda^2)^2} \bigg(1 + \frac{z}{\lambda} -4\omega^2\lambda^2(1 - \frac{z}{\lambda})\bigg)\bigg] \,.
\end{aligned}
\end{equation}
Therefore the GW is
\begin{equation}
\begin{aligned}
h^{(o)}_{ab}(\omega) = &  \sqrt{\frac{G_o}{G_s}}\exp\bigg[- \frac{\alpha e^{-z/\lambda}}{2(1+4\omega^2\lambda^2)^2} \bigg(1 + \frac{z}{\lambda} -4\omega^2\lambda^2(1 - \frac{z}{\lambda})\bigg)\bigg] \\
& \times \exp\bigg[- i k z\frac{\alpha e^{-z/\lambda}}{(1+4\omega^2\lambda^2)^2} \bigg(1 + \frac{2\lambda}{z} + 4\omega^2\lambda^2\bigg) \bigg] h_{ab}^{(s)}(\omega) \,. \label{spesptsol}
\end{aligned}
\end{equation}
Furthermore if $\omega \lambda \gg 1$, the eq.\eqref{spesptsol} becomes
\begin{equation}
\begin{aligned}
h^{(o)}_{ab}(\omega) = &  \sqrt{\frac{G_o}{G_s}}\exp\bigg[\frac{\alpha}{8\omega^2\lambda^2} \bigg(1 - \frac{z}{\lambda}\bigg)e^{-z/\lambda} \bigg] \exp\bigg[- i k z\frac{\alpha e^{-z/\lambda}}{4\omega^2\lambda^2} \bigg] h_{ab}^{(s)}(\omega) \\
= & \sqrt{\frac{G_o}{G_s}}\exp\bigg[-\frac{\partial_z^2 \ln G_o}{8\lambda^2} \bigg] \exp\bigg[- i k z\frac{\alpha e^{-z/\lambda}}{4\omega^2\lambda^2} \bigg] h_{ab}^{(s)}(\omega) \,.
\end{aligned}
\end{equation}
\end{comment}
If $\phi = \phi(t)$, the new variable $h_{ij} = C_{ij}e^{ikz}H(t)$ obeys the relation from eq.\eqref{fereom}
\begin{align}
\partial_t^2 H + \partial_t\ln\phi \,\partial_t H  + k^2H= 0
\end{align}
\begin{comment}
This, in terms of eq.\eqref{lapsol} with $y(t) = -\frac{1}{2}\partial_t^2g(t)$, $g(t) = -\ln G$, also makes the GW having the following relation during propagation
\begin{equation}
\begin{aligned}
h^{(o)}_{ab}(\omega) = & \sqrt{\frac{G_o}{G_s}} \exp\bigg[-ik t\frac{\partial_t^2 \ln G_o}{4k^2} \bigg] h_{ab}^{(s)}(\omega) = \sqrt{\frac{G_o}{G_s}} \exp\bigg[-i\omega t\frac{\partial_t^2 \ln G_o}{4k^2} \bigg] h_{ab}^{(s)}(\omega)\,, \label{ferlapsol}
\end{aligned}
\end{equation}
We expect an dispersion relation $\omega = k + \varepsilon k^{(1)}$ both in eq.\eqref{fersptsol} and \eqref{ferlapsol} and the second order term is omit, actually the dispersion relations from the phase corrections are given by
\begin{align}
\omega = & k + \frac{\partial_z^2\ln G_o}{4k} \quad \bigg( = k + \frac{\nabla^2\ln G}{4k} ????? \bigg)\,, \\
\omega = & k - \frac{\partial_t^2\ln G_o}{4k}\,.
\end{align}
in eq.\eqref{fersptsol} and eq.\eqref{ferlapsol} respectively. The phase corrections in eq.\eqref{fersptsol} and \eqref{ferlapsol} are obvious depending on the second order derivative and proportional to $1/\omega$, while the amplitude correction depends on a factor $\sqrt{G_o/G_s}$.
\end{comment}

\section{Action method for Einstein-Hilbert action}
Using the \textit{xAct} module in Mathematica, we expand the Lagrangian of Einstein-Hilbert action up to second order
\begin{equation}
\begin{aligned}
\mathcal{L}_{\mathrm{GR}} = \sqrt{-g}R = & -\frac{1}{2}\partial_b h \partial^b h - 2\partial_a h^{ab}\partial_c h^c_b + 2\partial_b h\partial_c h^{bc} + h\partial_b\partial_c h^{bc} \\
& + 2h^{ab}(\partial_a\partial_b h-\partial_b\partial_c h^c_a - \partial_c\partial_b h^c_a + \square h_{ab})\\
& - h \square h - \partial_b h_{ac}\partial^c h^{ab} + \frac{3}{2}\partial_c h_{ab}\partial^c h^{ab}
\end{aligned}
\end{equation}
Using the transverse trace variable 
\begin{align}
\bar{h}_{ab} = h_{ab} - \frac{1}{2}\eta_{ab}h\,,~~h_{ab} = \bar{h}_{ab} - \frac{1}{2}\eta_{ab}\bar{h}
\end{align}
and the gauge condition
\begin{align}
\partial_a \bar{h}^{ab} = 0\,,~~\partial_ah^{ab} = \frac{1}{2} \partial^b h=-\frac{1}{2}\partial^b\bar{h}
\end{align}
it becomes 
\begin{equation}
\begin{aligned}
\sqrt{-g}R = & -\frac{1}{2}\partial_b h \partial^b h - 2\partial_a h^{ab}\partial_c h^c_b + 2\partial_b h\partial_c h^{bc} + h\partial_b\partial_c h^{bc} + 2h^{ab}(\partial_a\partial_b h-\partial_b\partial_c h^c_a - \partial_c\partial_b h^c_a + \square h_{ab})\\
& - h \square h - \partial_b h_{ac}\partial^c h^{ab} + \frac{3}{2}\partial_c h_{ab}\partial^c h^{ab} \\
= & -\frac{1}{2}\partial_bh\partial^b h - \frac{1}{2}\partial^bh\partial_bh + \partial_bh\partial^bh + \frac{1}{2} h\square h + 2h^{ab}\partial_a\partial_b h - 2h^{ab}\partial_b\partial_a h + h^{ab}\square h_{ab} - h\square h\\
& - \partial_b h_{ac}\partial^c h^{ab} + \frac{3}{2}\partial_c h_{ab}\partial^c h^{ab}\\
= & -\frac{1}{2} h\square h + h^{ab}\square h_{ab} - \partial_b h_{ac}\partial^c h^{ab} + \frac{3}{2}\partial_c h_{ab}\partial^c h^{ab} \\
= & -\frac{1}{2} \bar{h}\square \bar{h} + \bar{h}^{ab}\square \bar{h}_{ab} - \partial_b \bar{h}_{ac}\partial^c \bar{h}^{ab} - \frac{1}{4}\partial_b\bar{h} \partial^b\bar{h} + \frac{3}{2}\partial_c \bar{h}_{ab}\partial^c \bar{h}^{ab}
\end{aligned}
\end{equation}
where we used the relations
\begin{align}
h^{ab} \square h_{ab} = & 2(\bar{h}^{ab}-\frac{1}{2}\eta^{ab}\bar{h})\square (\bar{h}_{ab} - \frac{1}{2}\eta_{ab}\bar{h}) = 2\bar{h}^{ab} \square \bar{h}_{ab} - 2 \bar{h}\square \bar{h} + 2\bar{h}\square \bar{h} = \bar{h}^{ab} \square \bar{h}_{ab} \\
- h \square h = & -\bar{h}\square \bar{h}\\
- \partial_b h_{ac}\partial^c h^{ab} = & -\partial_b(\bar{h}_{ac} - \frac{1}{2}\eta_{ac}\bar{h})\partial^c (\bar{h}^{ab}-\frac{1}{2}\eta^{ab}\bar{h}) = - \partial_b \bar{h}_{ac}\partial^c \bar{h}^{ab} + \partial_b \bar{h}\partial_a\bar{h}^{ab} - \frac{1}{4}\partial_b\bar{h} \partial^b\bar{h} \\
= & - \partial_b \bar{h}_{ac}\partial^c \bar{h}^{ab} - \frac{1}{4}\partial_b \bar{h} \partial^b \bar{h} \\
\frac{3}{2}\partial_c h_{ab}\partial^c h^{ab}  = & \frac{3}{2}\partial_c (\bar{h}_{ab} - \frac{1}{2}\eta_{ab}\bar{h}) \partial^c (\bar{h}^{ab}-\frac{1}{2}\eta^{ab}\bar{h}) = \frac{3}{2}\partial_c \bar{h}_{ab}\partial^c \bar{h}^{ab} -\frac{3}{2}\partial_c \bar{h} \partial^c \bar{h} + \frac{3}{2}\partial_ch\partial^c \bar{h} = \frac{3}{2}\partial_c \bar{h}_{ab}\partial^c \bar{h}^{ab}
\end{align}
Similarly the action for the GW is
\begin{equation}
\begin{aligned}
S = & \frac{1}{16\pi G} \int d^4x\bigg[  -\frac{1}{2} \bar{h}\square \bar{h} + \bar{h}^{ab}\square \bar{h}_{ab} - \partial_b \bar{h}_{ac}\partial^c \bar{h}^{ab} - \frac{1}{4}\partial_b\bar{h} \partial^b\bar{h} + \frac{3}{2}\partial_c \bar{h}_{ab}\partial^c \bar{h}^{ab} \bigg] \\
= & \frac{1}{16\pi } \int d^4x ~ \phi\bigg[  -\frac{1}{2} \bar{h}\square \bar{h} + \bar{h}^{ab}\square \bar{h}_{ab} - \partial_b \bar{h}_{ac}\partial^c \bar{h}^{ab} - \frac{1}{4}\partial_b\bar{h} \partial^b\bar{h} + \frac{3}{2}\partial_c \bar{h}_{ab}\partial^c \bar{h}^{ab} \bigg] \\
\end{aligned}
\end{equation}
We work in TT gauge
\begin{equation}
\begin{aligned}
S = & \frac{1}{16\pi } \int d^4x ~ \phi\bigg[ \bar{h}^{ij}\square \bar{h}_{ij} - \partial_j \bar{h}_{ik}\partial^k \bar{h}^{ij} + \frac{3}{2}\partial_c \bar{h}_{ij}\partial^c \bar{h}^{ij} \bigg] \\
\end{aligned}
\end{equation}
and EOM becomes
\begin{equation}
\begin{aligned}
\phi\square \bar{h}_{ij} + \square (\phi \bar{h}_{ij}) + 2\partial^k(\phi\partial_j \bar{h}_{ik}) - \frac{3}{2}\partial^a (\phi \partial_a \bar{h}_{ij}) = 0
\end{aligned}
\end{equation}
or
\begin{equation}
\begin{aligned}
\square \bar{h}_{ij} + \bigg(\frac{2}{\phi}\square \phi \bigg)\bar{h}_{ij} + \partial^a \ln\phi \partial_a\bar{h}_{ij} + 4\partial^k \ln\phi\partial_j \bar{h}_{ik} = 0
\end{aligned}
\end{equation}
The first case assumed as $\phi = \phi(\vec{r})$, we obtain the differential equations of $\bar{h}_{ij}$
\begin{equation}
\begin{aligned}
\square \bar{h}_{ij} + \bigg(\frac{2}{\phi}\nabla^2 \phi\bigg) \bar{h}_{ij} + \partial^k \ln\phi \partial_k \bar{h}_{ij} + 4\partial^k \ln\phi\partial_j \bar{h}_{ik} = 0
\end{aligned}
\end{equation}
We then separate the variables as $\bar{h}_{ij} = H_{ij}e^{-i\omega t}$ so the spatial differential equations become
\begin{equation}
\begin{aligned}
\nabla^2H_{ij} + \bigg[ \omega^2 + \frac{2}{\phi}\nabla^2 \phi\bigg]H_{ij} + \partial^k \ln\phi \partial_k H_{ij} + 4\partial^k \ln\phi\partial_j H_{ik} = 0
\end{aligned}
\end{equation}
Because the GW travels along $z$-axis, the $H_{ij}$ is the function of $z$ only,
\begin{equation}
\begin{aligned}
    \partial_z^2 H_{ij} + \bigg[ \omega^2 + \frac{2}{\phi}\partial_z^2 \phi \bigg]H_{ij} + \partial_z \ln\phi \partial_z H_{ij} + 4\delta^z_j\partial^k \ln\phi\partial_z H_{ik} = 0
\end{aligned}
\end{equation}
In TT gauge, $H_{ij}$ furthermore only has $x,y$ components so the equations become
\begin{equation}
\begin{aligned}
    \partial_z^2 h+ \partial_z \ln\phi\,\partial_z h + \bigg[ \omega^2 + \frac{2}{\phi}\partial_z^2 \phi \bigg]h  = 0 \label{desspteq1}
\end{aligned}
\end{equation}
where $h$ represents $h_+$ or $h_\times$. Approximately, this system has a frequency shift $\frac{1}{\omega\phi}\partial_z^2\phi$.\begin{comment} This equation has the same form to the eq.\eqref{dessoleq} with $f(z) = \frac{2}{\phi}\nabla^2 \phi \approx 2\nabla^2\ln \phi$, $g(z) = \ln\phi$. Therefore the forward wave solution to eq.\eqref{desspteq1} is
\begin{equation}
\begin{aligned}
h^{(o)}_{ab}(\omega) = & \sqrt{\frac{G_o}{G_s}} \exp\bigg[iz \frac{ 4\nabla^2\ln \phi -\partial_z^2 \ln\phi}{4\omega} \bigg] h_{ab}^{(s)}(\omega) \\
= & \sqrt{\frac{G_o}{G_s}} \exp\bigg[-ikz \frac{ 4\nabla^2\ln G_o -\partial_z^2 \ln G_o}{4\omega^2} \bigg] h_{ab}^{(s)}(\omega) \,.
\label{ehsptsol}
\end{aligned}
\end{equation}
\textcolor{red}{The general form is ??}
\begin{equation}
h^{(o)}_{ab}(\omega) = \sqrt{\frac{G_o}{G_s}} \exp\bigg[ \frac{3\nabla^2 \ln G_o}{8\omega^2}\bigg] \exp\bigg[-i\frac{ 3\nabla^2 \ln G_o}{4\omega^2} \vec{k}\cdot\vec{r} \bigg] h_{ab}^{(s)}(\omega) ??
\end{equation}

Now we consider the specific form of gravitational constant as
\begin{equation}
G = G_s(1+\alpha(1+z/\lambda) e^{-z/\lambda})\,.
\end{equation}
Again using the two variable expansion method and matching the eq.\eqref{heq} with
\begin{align}
y(z) = & 2\nabla^2\ln\phi - \frac{1}{2}\partial_z^2\ln\phi = \frac{\alpha}{2\lambda^2}\bigg( 11 - \frac{3z}{\lambda} \bigg) e^{-z/\lambda}\,,\\
g(z) = & \ln\phi = -\ln G \approx - \alpha\bigg(1+\frac{z}{\lambda}\bigg)e^{-z/\lambda}\,,
\end{align}
the solution is 
\begin{equation}
\begin{aligned}
h = & A\cos(\omega z) + B\sin(\omega z) \\
& + \frac{\alpha e^{-z/\lambda}}{2(1+4\omega^2\lambda^2)^2}\bigg[ 5 - \frac{3z}{\lambda} + \bigg(44 - \frac{12z}{\lambda}\bigg) \lambda^2\omega^2 \bigg]\big[ A\cos(\omega z) + B\sin(\omega z) \big]  \\
& + \frac{\alpha e^{-z/\lambda}}{(1+4\omega^2\lambda^2)^2}\omega z \bigg[ 3 - \frac{2\lambda}{z} + \bigg(12 - \frac{16\lambda}{z} \bigg) \lambda^2\omega^2 \bigg] \big[A\sin(\omega z) - B\cos(\omega z)\big]
\end{aligned}
\end{equation}
The in-going wave condition $h = \mathcal{A}e^{i\omega z}$ requires $B = iA, A = \mathcal{A}$ and $h$ becomes
\begin{equation}
\begin{aligned}
h = & \mathcal{A}e^{i\omega z}\bigg\{ 1 + \frac{\alpha e^{-z/\lambda}}{2(1+4\omega^2\lambda^2)^2} \bigg[ 5 - \frac{3z}{\lambda} + \bigg(44 - \frac{12z}{\lambda}\bigg) \lambda^2\omega^2 \bigg] \\
&  - i \frac{\alpha e^{-z/\lambda}}{(1+4\omega^2\lambda^2)^2}\omega z \bigg[ 3 - \frac{2\lambda}{z} + \bigg(12 - \frac{16\lambda}{z} \bigg) \lambda^2\omega^2 \bigg] \bigg\} \\
= & \mathcal{A}e^{i\omega z} \exp\bigg\{ \frac{\alpha e^{-z/\lambda}}{2(1+4\omega^2\lambda^2)^2} \bigg[ 5 - \frac{3z}{\lambda} + \bigg(44 - \frac{12z}{\lambda}\bigg) \lambda^2\omega^2 \bigg] \bigg\} \\
& \times \exp\bigg\{ i \frac{\alpha e^{-z/\lambda}}{(1+4\omega^2\lambda^2)^2}\omega\lambda \bigg[ 2 -\frac{3z}{\lambda} + \bigg(16 - \frac{12z}{\lambda} \bigg) \lambda^2\omega^2 \bigg] \bigg\} \,.
\end{aligned}
\end{equation}
Therefore the GW is
\begin{equation}
\begin{aligned}
h^{(o)}_{ab}(\omega) = &  \sqrt{\frac{G_o}{G_s}} \exp\bigg\{ \frac{\alpha e^{-z/\lambda}}{2(1+4\omega^2\lambda^2)^2} \bigg[ 5 - \frac{3z}{\lambda} + \bigg(44 - \frac{12z}{\lambda}\bigg) \lambda^2\omega^2 \bigg] \bigg\} \\
& \times \exp\bigg\{ i \frac{\alpha e^{-z/\lambda}}{(1+4\omega^2\lambda^2)^2}\omega\lambda \bigg[ 2 -\frac{3z}{\lambda} + \bigg(16 - \frac{12z}{\lambda} \bigg) \lambda^2\omega^2 \bigg] \bigg\} h^{(s)}_{ab}(\omega) \,. \label{ehspesptsol}
\end{aligned}
\end{equation}
Furthermore if $\omega \lambda \gg 1$, the eq.\eqref{ehspesptsol} becomes
\begin{equation}
\begin{aligned}
h^{(o)}_{ab}(\omega) = & \sqrt{\frac{G_o}{G_s}} \exp\bigg[ \frac{\alpha e^{-z/\lambda}}{8\omega^2\lambda^2} \bigg(11 - \frac{3z}{\lambda}\bigg)  \bigg] \exp\bigg[ i \frac{\alpha e^{-z/\lambda}}{\omega\lambda} \bigg(1 - \frac{3z}{4\lambda} \bigg) \bigg] h_{ab}^{(s)}(\omega) \\
= & \sqrt{\frac{G_o}{G_s}} \exp\bigg[ -\frac{1}{4\omega^2} \bigg(2\nabla^2\ln G - \frac{1}{2}\partial_z^2\ln G\bigg)  \bigg] \exp\bigg[ i \frac{\alpha e^{-z/\lambda}}{\omega\lambda} \bigg(1 - \frac{3z}{4\lambda} \bigg) \bigg] h_{ab}^{(s)}(\omega) \,.
\end{aligned}
\end{equation}
\end{comment}


If the dependence becomes $\phi = \phi(t)$, the EOM becomes
\begin{equation}
\begin{aligned}
\frac{1}{2}\square \bar{h}_{ij} + \frac{\bar{h}_{ij}}{\phi}\square \phi + \frac{1}{2} \partial^a \ln\phi \partial_a\bar{h}_{ij} + 2\partial^k \ln\phi\partial_j \bar{h}_{ik} = 0
\end{aligned}
\end{equation}
or after variable separation $\bar{h}_{ij} = C_{ij}e^{ikz}H(t)$
\begin{equation}
\begin{aligned}
\partial_t^2H + \partial_t\ln\phi\,\partial_tH + \left[k^2 + \frac{2}{\phi}\partial_t^2 \phi\right] H = 0
\end{aligned}
\end{equation}
\begin{comment}
This, in terms of eq.\eqref{lapsol} with $y(t) = 2\partial_t^2\phi - \frac{1}{2}\partial_t^2\ln\phi = \frac{3}{2}\partial_t^2\ln\phi = -\frac{3}{2}\partial_t^2\ln G$, $g(t) = -\ln G$, also makes the forward GW having the following relation during propagation
\begin{equation}
\begin{aligned}
h^{(o)}_{ab}(\omega) = & \sqrt{\frac{G_o}{G_s}} \exp\bigg[ik t\frac{3\partial_t^2 \ln G_o}{4k^2} \bigg] h_{ab}^{(s)}(\omega) = \sqrt{\frac{G_o}{G_s}} \exp\bigg[i\omega t\frac{3\partial_t^2 \ln G_o}{4k^2} \bigg] h_{ab}^{(s)}(\omega)\,. \label{ehlapsol}
\end{aligned}
\end{equation}
The dispersion relations from the phase corrections are given by
\begin{align}
\omega = & k - \frac{ 4\nabla^2\ln G_o -\partial_z^2 \ln G_o}{4k} \quad \bigg( = k + \frac{ 3\nabla^2 \ln G_o}{4k}????? \bigg) \,, \\
\omega = & k + \frac{ 3\partial_t^2 \ln G_o}{4k} \,.
\end{align}
in eq.\eqref{ehsptsol} and eq.\eqref{ehlapsol} respectively. These dispersion relation gives rise to the derivate from light speed by
\begin{align}
|\nabla^2\ln G| < \frac{16\pi^2}{3} f^2 \Delta c\,, \\
|\partial_t^2\ln G| < \frac{16\pi^2}{3} f^2 \Delta c\,.
\end{align}
\end{comment}

\section{Appendix A: Solving wave equation}
\label{sec:appa}
In this appendix, we solve the wave equation. By expressing $h_{ij}$ as a Fourier transform, the
wave equation for each Fourier amplitude becomes the following general linear form,
\begin{equation}
    \frac{d^2}{d z^2}H(z)+2p(z)\frac{d}{d z}H(z)+\left[\omega^2+q(z)\right]H(z)=0.\label{dessoleq}
\end{equation}

For any $H(z)$, there are real function $A(z)$ and $\Phi(z)$ which satisfy
\begin{equation}\label{HAphi}
    H=Ae^{i\Phi}.
\end{equation}
We plug eq.\eqref{HAphi} into eq.\eqref{dessoleq} and then divide both sides of the equation by $e^{i\Phi}$. From the real and imaginary part of the result, we achieve two equations,
\begin{equation}\label{rpart}
    \frac{d^2 A}{d z^2}+2p\frac{d A}{d z}+\left[\omega^2\left(1-\frac{k^2}{\omega^2}\right)+q\right]A=0,
\end{equation}
\begin{equation}\label{ipart}
    2\frac{d A}{d z}k+A\frac{d k}{d z}+2pAk=0,
\end{equation}
where $k=\frac{d \Phi}{d z}$. It is no matter for us to suppose that $A>0$ and $k>0$, and then
\begin{equation}
    2\frac{1}{A}\frac{d A}{d z}+\frac{1}{k}\frac{d k}{d z}+2p=0,
\end{equation}
therefore
\begin{equation}\label{Apk}
    A\propto e^{-\int p\,dz}k^{-1/2}.
\end{equation}
We plug eq.\eqref{Apk} into eq.\eqref{rpart} and then achieve
\begin{equation}\label{equK0}
    \frac{d^2 K}{d z^2}-\left(\frac{1}{\Gamma}\frac{d^2\Gamma}{d z^2}-q\right)K+\omega^2K(1-K^{-4})=0,
\end{equation}
where $\Gamma=e^{\int p \,dz}$ and $K=(k/\omega)^{-1/2}$. If $A=\check{A}$ and $k=+\check{k}$ satisfy eq.\eqref{rpart} and eq.\eqref{ipart}, $A=\check{A}$ and $k=-\check{k}$ will satisfy eq.\eqref{rpart} and eq.\eqref{ipart} as well. This means that since eq.\eqref{dessoleq} is linear, the general solution of eq.\eqref{dessoleq} can be expressed as
\begin{equation}
    H=C_+e^{-\int p\,dz}\check{K}e^{+i\omega\int  \check{K}^{-2}\,\d z}+C_-e^{-\int p\,dz}\check{K}e^{-i\omega\int  \check{K}^{-2}\,\d z},
\end{equation}
where $\check{K}$ is a particular solution of eq.\eqref{equK0}.

Now we need to find a particular solution of eq.\eqref{equK0}. We let $\Xi=\frac{1}{\Gamma}\frac{d^2\Gamma}{d z^2}-q$ and make $\omega=1$, which means that we are now using natural units where $c=1$ and $\omega=1$, then
\begin{equation}\label{equK}
    \frac{d^2 K}{d z^2}+K[(1-\Xi)-K^{-4}]=0.
\end{equation}
If $\Xi=\text{const}$, it is easy to find a particular solution
\begin{equation}\label{K0}
    K=(1-\Xi)^{-1/4}=1+\frac{1}{4}\Xi+\frac{5}{32}\Xi^2+O(\Xi^3),
\end{equation}
which means
\begin{equation}
    k=(1-\Xi)^{1/2}=1-\frac{1}{2}\Xi-\frac{1}{8}\Xi^2+O(\Xi^3),
\end{equation}
and this is similar to which of damped oscillations. If $\Xi\neq\text{const}$, we are interested in a special situation that $\Xi(z)=\kappa^2\tilde{\Xi}(\tilde{z})$, where $\tilde{z}=\kappa z$ and $\kappa$ is a small amount. In this situation,
\begin{equation}\label{equKs}
    K^3\frac{d^2 K}{d \tilde{z}^2}\kappa^2-K^4\tilde{\Xi}(\tilde{z})\kappa^2+K^4-1=0.
\end{equation}
We suppose that
\begin{equation}\label{K}
    K=\sum_{n=0}^\infty K_n(\tilde{z})\kappa^{2n},
\end{equation}
and if we plug eq.\eqref{K} into eq.\eqref{equKs}, we will maintain a power series of $\kappa^2$ which is equal to $0$. We make each of the coefficients of this power series equal to $0$, for example,
\begin{equation}
    K_0^4-1=0,
\end{equation}
\begin{equation}
    K_0^3K_0''-K_0^4\tilde{\Xi}+4K_0^3K_1=0,
\end{equation}
\begin{equation}
    (K_0^3K_1''+3K_0^2K_1K_0'')-4K_0^3K_1\tilde{\Xi}+(4K_0^3K_2+6K_0^2K_1^2)=0.
\end{equation}
Finally, by solving all of these equations in order, we will find each of the coefficients in eq.\eqref{K}, for example,
\begin{equation}
    K_0=1,
\end{equation}
\begin{equation}
    K_1=\frac{1}{4}\tilde{\Xi},
\end{equation}
\begin{equation}
    K_2=\frac{5}{32}\tilde{\Xi}^2-\frac{1}{16}\frac{d^2\tilde{\Xi}}{d\tilde{z}^2},
\end{equation}
and these coefficients will make our particular solution of $K$ similar to eq.\eqref{K0}.

\begin{comment}
\begin{equation}
    4K_0^3K_1\kappa=0
\end{equation}
\begin{equation}
    K_1=0
\end{equation}
\begin{equation}
    [K_0^3K_0''-K_0^4\tilde{\Xi}(\tilde{z})+(4K_0^3K_2+6K_0^2K_1^2)]\kappa^2=0
\end{equation}
\begin{equation}
    K_2=\frac{1}{4}\tilde{\Xi}(\tilde{z})
\end{equation}
\begin{equation}
    [(K_0^3K_1''+3K_0^2K_1K_0)-4K_0^3K_1\Xi(\tilde{z})+(4K_0^3K_3+12K_0^2K_1K_2+4K_0K_1^3)]\kappa^3=0
\end{equation}
\begin{equation}
    K_3=0
\end{equation}

\begin{equation}
    K^{3}\frac{d^2 K}{d z^2}+K^{4}[1-\Xi(\frac{z}{l})\frac{1}{l^2}]-1=0
\end{equation}
\begin{equation}
    K=\sum_{n=0}^\infty K_n(\frac{z}{l})\frac{1}{l^n}
\end{equation}
\begin{equation}
    K_0(\frac{z}{l})^4-1=0
\end{equation}
\begin{equation}
    K_0=1
\end{equation}
\begin{equation}
    4K_0(\frac{z}{l})^3K_1(\frac{z}{l})\frac{1}{l}=0
\end{equation}
\begin{equation}
    K_1=0
\end{equation}
\begin{equation}
    K_0(\frac{z}{l})^3K_0''(\frac{z}{l})\frac{1}{l^2}+[4K_0(\frac{z}{l})^3K_2(\frac{z}{l})+6K_0(\frac{z}{l})^2K_1(\frac{z}{l})^2]\frac{1}{l^2}-K_0(\frac{z}{l})^4\Xi(\frac{z}{l})\frac{1}{l^2}=0
\end{equation}
\begin{equation}
    K_2(\frac{z}{l})=\frac{1}{4}\Xi(\frac{z}{l})
\end{equation}
\begin{equation}
    [K_0(\frac{z}{l})^3K_1''(\frac{z}{l})+3K_0(\frac{z}{l})^2K_1(\frac{z}{l})K_0''(\frac{z}{l})]\frac{1}{l^3}+[4K_0(\frac{z}{l})^3K_3(\frac{z}{l})+12K_0(\frac{z}{l})^2K_1(\frac{z}{l})K_2(\frac{z}{l})+4K_0(\frac{z}{l})K_1(\frac{z}{l})^3]\frac{1}{l^3}-4K_0(\frac{z}{l})^3K_1(\frac{z}{l})\Xi(\frac{z}{l})\frac{1}{l^3}=0
\end{equation}
\begin{equation}
    K_3(\frac{z}{l})=0
\end{equation}
\begin{equation}
    [K_0(\frac{z}{l})^3K_2''(\frac{z}{l})+3K_0(\frac{z}{l})^2K_1(\frac{z}{l})K_1''(\frac{z}{l})+[3K_0(\frac{z}{l})K_1(\frac{z}{l})^2+3K_0(\frac{z}{l})^2K_2(\frac{z}{l})]K_0''(\frac{z}{l})]\frac{1}{l^4}+[4K_0(\frac{z}{l})^3K_4(\frac{z}{l})+12K_0(\frac{z}{l})^2K_1(\frac{z}{l})K_3(\frac{z}{l})+6K_0(\frac{z}{l})^2K_2(\frac{z}{l})^2+12K_0(\frac{z}{l})K_1(\frac{z}{l})^2K_2(\frac{z}{l})+K_1(\frac{z}{l})^4]\frac{1}{l^4}-[4K_0(\frac{z}{l})^3K_2(\frac{z}{l})+6K_0(\frac{z}{l})^2K_1(\frac{z}{l})^2]\Xi(\frac{z}{l})\frac{1}{l^4}=0
\end{equation}
\begin{equation}
    [K_2''(\frac{z}{l})]\frac{1}{l^4}+[4K_4(\frac{z}{l})+6K_2(\frac{z}{l})^2]\frac{1}{l^4}-[4K_2(\frac{z}{l})]\Xi(\frac{z}{l})\frac{1}{l^4}=0
\end{equation}
\begin{equation}
    K_2''(\frac{z}{l})+4K_4(\frac{z}{l})-\frac{5}{8}\Xi(\frac{z}{l})^2=0
\end{equation}

\begin{equation}
    K=?(1-\Xi)^{-1/4}
\end{equation}
\begin{equation}
    K'=?'(1-\Xi)^{-1/4}+(1/4)?(1-\Xi)^{-5/4}\Xi'
\end{equation}
\begin{equation}
    K''=?''(1-\Xi)^{-1/4}+(1/2)?'(1-\Xi)^{-5/4}\Xi'+(5/16)?(1-\Xi)^{-9/4}(\Xi')^2+(1/4)?(1-\Xi)^{-5/4}\Xi''
\end{equation}
\begin{equation}
    K[(1-\Xi)-K^{-4}]=?(1-\Xi)^{3/4}(1-?^{-4})
\end{equation}

\begin{equation}
    (G'/G)^2,\quad G''/G
\end{equation}
\begin{equation}
    G = G_0[1+\alpha(z/\lambda+1) e^{-z/\lambda}]
\end{equation}
\begin{equation}
    G' = (G_0/\lambda)[-\alpha(z/\lambda) e^{-z/\lambda}]
\end{equation}
\begin{equation}
    G'' = (G_0/\lambda^2)[\alpha(z/\lambda-1) e^{-z/\lambda}]
\end{equation}

\begin{equation}
    (\phi'/\phi)^2,\quad\phi''/\phi
\end{equation}
\begin{equation}
    \phi=G^{-1}
\end{equation}
\begin{equation}
    \phi'=-G^{-2}G'
\end{equation}
\begin{equation}
    \phi'=2G^{-3}(G')^2-G^{-2}G''
\end{equation}

\begin{equation}
    p=\frac{1}{2}\frac{\d}{\d z}\ln\phi
\end{equation}
\begin{equation}
    \Gamma=(\phi/\phi_0)^{1/2}
\end{equation}
\begin{equation}
    \Gamma'=\frac{1}{2}(\phi/\phi_0)^{-1/2}(\phi'/\phi_0)
\end{equation}
\begin{equation}
    \Gamma''=-\frac{1}{4}(\phi/\phi_0)^{-3/2}(\phi'/\phi_0)^2+\frac{1}{2}(\phi/\phi_0)^{-1/2}(\phi''/\phi_0)
\end{equation}
\begin{equation}
    \Gamma''/\Gamma=-\frac{1}{4}(\phi/\phi_0)^{-2}(\phi'/\phi_0)^2+\frac{1}{2}(\phi/\phi_0)^{-1}(\phi''/\phi_0)
\end{equation}
\begin{equation}
    q=2\phi''/\phi
\end{equation}

\begin{equation}
    G = G_0[1+\alpha(z/\lambda+1) e^{-z/\lambda}]
\end{equation}
\begin{equation}
    G' = (G_0/\lambda)[-\alpha(z/\lambda) e^{-z/\lambda}]
\end{equation}
\begin{equation}
    G'' = (G_0/\lambda^2)[\alpha(z/\lambda-1) e^{-z/\lambda}]
\end{equation}
\begin{equation}
    p=-\frac{1}{2}\frac{\d}{\d z}\ln G
\end{equation}
\begin{equation}
    \Gamma=e^{\int p \,\d z}=e^{\int -\frac{1}{2}\frac{\d}{\d z}\ln G \,\d z}=e^{ -\frac{1}{2}\ln(G/G_0)}=(G/G_0)^{-\frac{1}{2}}
\end{equation}
\begin{equation}
    \Gamma'=-\frac{1}{2}(G/G_0)^{-\frac{3}{2}}(G'/G_0)
\end{equation}
\begin{equation}
    \Gamma''=\frac{3}{4}(G/G_0)^{-\frac{5}{2}}(G'/G_0)^2-\frac{1}{2}(G/G_0)^{-\frac{3}{2}}(G''/G_0)
\end{equation}
\begin{equation}
    \Gamma''/\Gamma=\frac{3}{4}(G/G_0)^{-2}(G'/G_0)^2-\frac{1}{2}(G/G_0)^{-1}(G''/G_0)
\end{equation}
\begin{equation}
    \frac{3}{4}\frac{[\alpha(z/\lambda) e^{-z/\lambda}]^2}{[1+\alpha(z/\lambda+1) e^{-z/\lambda}]^2}-\frac{1}{2}\frac{\alpha(z/\lambda-1) e^{-z/\lambda}}{1+\alpha(z/\lambda+1) e^{-z/\lambda}}
\end{equation}

\begin{equation}
    \frac{d^2 K}{d z^2}+\omega^2K(1-\frac{\Xi}{\omega^2}-K^{-4})=0
\end{equation}
\begin{equation}
    K = (1-\frac{\Xi}{\omega^2})^{-1/4}
\end{equation}

\begin{equation}
    \frac{d^2 \Delta K}{d z^2}+\omega^2(1+\Delta K)[4\Delta K+\sum_{n=2}^{\infty}(-1)^{n+1}C^2_{n+2}{\Delta K}^n-\frac{\Xi}{\omega^2}]=0
\end{equation}
\begin{equation}
    \frac{d^2 \Delta K}{d z^2}+\omega^2[4\Delta K-\frac{\Xi}{\omega^2}]=0
\end{equation}
\begin{equation}
    \Delta K=\frac{\Xi}{4\omega^2}
\end{equation}

\begin{equation}
    \Delta K=\left[\Delta K|_{z=0}-\int_{0}^z\frac{\Xi\sin(2\omega z)}{2\omega}\,d z\right]\cos(2\omega z)+\left[\frac{1}{2\omega}\frac{d \Delta K}{d z}|_{z=0}+\int_{0}^z\frac{\Xi\cos(2\omega z)}{2\omega}\,d z\right]\sin(2\omega z)
\end{equation}
\begin{equation}
    \Delta K=-\left[\int_{0}^z\frac{\Xi\sin(2\omega z)}{2\omega}\,d z\right]\cos(2\omega z)+\left[\int_{0}^z\frac{\Xi\cos(2\omega z)}{2\omega}\,d z\right]\sin(2\omega z)
\end{equation}
\begin{equation}
    \Delta K=\frac{\Xi}{(2\omega)^2}[1-\cos(2\omega z)]-\left[\int_{0}^z\frac{\frac{d\Xi}{d z}\cos(2\omega z)}{(2\omega)^2}\,d z\right]\cos(2\omega z)-\left[\int_{0}^z\frac{\frac{d\Xi}{d z}\sin(2\omega z)}{(2\omega)^2}\,d z\right]\sin(2\omega z)
\end{equation}
\end{comment}



\bibliography{reference}

\end{document}
