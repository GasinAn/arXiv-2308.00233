\documentclass[
 jor,
 amsmath,amssymb,preprint,
 %reprint,
%author-year,%
%author-numerical,%
]{revtex4-2}

\usepackage{subfig}
\usepackage{color}
\usepackage{graphicx}% Include figure files
\usepackage{dcolumn}% Align table columns on decimal point
\usepackage{bm}% bold math
%\usepackage[mathlines]{lineno}% Enable numbering of text and display math
%\linenumbers\relax % Commence numbering lines
\usepackage{hyperref}
\usepackage{verbatim}

\def\d{\mathrm{d}}
\def\p{\partial}

\begin{document}

\preprint{AIP/123-QED}

\title[Probe Varying Gravitational Constant via GW]{Probe Varying Gravitational Constant via GW}% Force line breaks with \\



\author{Bing Sun}
\affiliation{%
CAS Key Laboratory of Theoretical Physics, Institute of Theoretical Physics, Chinese Academy of Sciences, Beijing 100190, China%\\This line break forced% with \\
}%

\date{\today}% It is always \today, today,
             %  but any date may be explicitly specified

\begin{abstract}
In this note, we give the effect of a varying gravitational constant $G$ on the gravitational waves (GWs) during propagation, which finally gives a very simple amplitude modification $\sqrt{\frac{G^{(o)}}{G^{(s)}}}$ and a phase correction proportional to ${\partial_t^2}\ln G$ ($\nabla^2 \ln G$) and $1/\omega$ for observation position.
%
\end{abstract}

\keywords{gravitational constant, gravitational waves}

\maketitle

\section{Action method analogous to Electromagnetic Field}
The Fierz-Pauli action for the field variable $h_{ab}$ is given by~\cite{roy2020probing}
\begin{equation}
\begin{aligned}
S = & \frac{1}{64 \pi G} \int d^4 x\left(-\partial_a h_{b c} \partial^a h^{b c}+\partial_a h_b^b \partial^a h_c^c-2 \partial_a h_c^a \partial^c h_b^b+2 \partial_a h_c^a \partial_b h^{b c}\right)
\end{aligned}
\end{equation}
We set gravitational constant $G$ as a function in a form $\phi^{-1}$ so that the action becomes
\begin{equation}
\begin{aligned}
S = & \frac{1}{64 \pi} \int d^4 x~\phi\left(-\partial_a h_{b c} \partial^a h^{b c}+\partial_a h_b^b \partial^a h_c^c-2 \partial_a h_c^a \partial^c h_b^b+2 \partial_a h_c^a \partial_b h^{b c}\right) \label{origact}
\end{aligned}
\end{equation}
In the Lorentz gauge of transverse trace variable $\bar{h}_{ab} = h_{ab} - \frac{1}{2}\eta_{ab}h$, $\partial^b \bar{h}_{ab} = 0$, the action \eqref{origact} has a simple form
\begin{equation}
\begin{aligned}
S = & \frac{1}{64 \pi} \int d^4 x~\phi \left( -\partial_a\bar{h}_{bc}\partial^a\bar{h}^{bc} + \frac{1}{2}\partial_c\bar{h}\partial^c\bar{h} \right)
\end{aligned}
\end{equation}
The equation of motion (EOM) for this action becomes
\begin{align}
\square h_{ab} + \partial_ch_{ab}\partial^c\ln\phi = 0 \label{fereom}
\end{align}
where $h_{ab} = \bar{h}_{ab} - \frac{1}{2} \eta_{ab}\bar{h}$. We separate the variable as $h_{ab} = H_{ab}e^{-i\omega t}$
\begin{equation}
\begin{aligned}
\nabla^2 H_{ab} + \omega^2 H_{ab} + i\omega \partial_t\ln \phi H_{ab} + \delta^{ij} \partial_j\ln \phi \partial_iH_{ab} = 0
\end{aligned}
\end{equation}
If the GW travels along $z$-axis in a transverse-traceless (TT) gauge, $H_{ij} = H_{ij}(z)$ and the equation becomes
\begin{align}
\partial_z^2 H_{ij} + \omega^2 H_{ij} + i\omega \partial_t\ln \phi H_{ij} + \partial_z\ln \phi \partial_zH_{ij} = 0 \label{ferorg}
\end{align}
Furthermore, if $\phi = \phi(\vec{r})$, the expression eq.\eqref{ferorg} is solvable
\begin{equation}
\begin{aligned}
\partial_z^2 H_{ij} + \partial_z \ln\phi\,\partial_z H_{ij} + \omega^2 H_{ij} = 0 \,,
\end{aligned}
\end{equation}
\begin{comment}
giving rise to, in terms of eq.\eqref{sptsol} in Appendix \ref{sec:appa} with $y(z) = -\frac{1}{2}\partial_z^2g(z)$, $g(z) = \ln\phi$,
\begin{equation}
\begin{aligned}
h^{(o)}_{ab}(\omega) = & \sqrt{\frac{G_o}{G_s}} \exp\bigg[ - \frac{i z\partial_z^2 \ln\phi}{4\omega} \bigg] h_{ab}^{(s)}(\omega) = \sqrt{\frac{G_o}{G_s}} \exp\bigg[i\omega z\frac{\partial_z^2 \ln G_o}{4\omega^2} \bigg] h_{ab}^{(s)}(\omega) \,. \label{fersptsol}
\end{aligned}
\end{equation}
\textcolor{red}{The general form is ??}
\begin{equation}
h^{(o)}_{ab}(\omega) = \sqrt{\frac{G_o}{G_s}} \exp\bigg[i\frac{ \nabla^2 \ln G_o}{4\omega^2} \vec{k}\cdot\vec{r} \bigg] h_{ab}^{(s)}(\omega) ??
\end{equation}

Now we consider a special case
\begin{equation}
G = G_0(1+\alpha(1+r/\lambda) e^{-r/\lambda})
\end{equation}
or we use the symbol $z$
\begin{equation}
G = G_s(1+\alpha(1+z/\lambda) e^{-z/\lambda})
\end{equation}
where $z$ in radiation frame is equivalent to $r$ in source frame. Thus the function $y(z)$ is written as
\begin{equation}
y(z) = \frac{1}{2} \partial_z^2 \ln G = -\frac{\alpha}{2\lambda^2}e^{-z/\lambda} \bigg(1 - \frac{z}{\lambda} \bigg) \,.
\end{equation}
and substituting it into eq.\eqref{desfirorddiffeq} with $\xi = z$ we obtain
\begin{equation}
\begin{aligned}
h = & A\cos(\omega z) + B \sin(\omega z) \\
& + \frac{\alpha e^{-z/\lambda}}{2(1+4\omega^2\lambda^2)^2} \bigg[ 2\omega z(1 + \frac{2\lambda}{z} + 4\omega^2\lambda^2)(A\sin(\omega z)-B\cos(\omega z)) \\
& - (1 + \frac{z}{\lambda} -4\omega^2\lambda^2(1 - \frac{z}{\lambda}))(A\cos(\omega z)+B\sin(\omega z))\bigg] \,.
\end{aligned}
\end{equation}
The in-going wave condition $h = \mathcal{A}e^{i\omega z}$ requires $B = iA, A = \mathcal{A}$ and $h$ becomes
\begin{equation}
\begin{aligned}
h = & \mathcal{A}e^{i\omega z} + \mathcal{A}\frac{\alpha e^{-z/\lambda}}{2(1+4\omega^2\lambda^2)^2} \bigg[ -2i\omega z(1 + \frac{2\lambda}{z} + 4\omega^2\lambda^2)e^{i\omega z} - (1 + \frac{z}{\lambda} -4\omega^2\lambda^2(1 - \frac{z}{\lambda}))e^{i\omega z}\bigg] \\
= & \mathcal{A}e^{i\omega z} \bigg[1 - i\omega z\frac{\alpha e^{-z/\lambda}}{(1+4\omega^2\lambda^2)^2} \bigg(1 + \frac{2\lambda}{z} + 4\omega^2\lambda^2\bigg) - \frac{\alpha e^{-z/\lambda}}{2(1+4\omega^2\lambda^2)^2} \bigg(1 + \frac{z}{\lambda} -4\omega^2\lambda^2(1 - \frac{z}{\lambda})\bigg)  \bigg] \\
= & \mathcal{A}e^{i\omega z} \exp\bigg[- i\omega z\frac{\alpha e^{-z/\lambda}}{(1+4\omega^2\lambda^2)^2} \bigg(1 + \frac{2\lambda}{z} + 4\omega^2\lambda^2\bigg) \bigg] \exp\bigg[- \frac{\alpha e^{-z/\lambda}}{2(1+4\omega^2\lambda^2)^2} \bigg(1 + \frac{z}{\lambda} -4\omega^2\lambda^2(1 - \frac{z}{\lambda})\bigg)\bigg] \,.
\end{aligned}
\end{equation}
Therefore the GW is
\begin{equation}
\begin{aligned}
h^{(o)}_{ab}(\omega) = &  \sqrt{\frac{G_o}{G_s}}\exp\bigg[- \frac{\alpha e^{-z/\lambda}}{2(1+4\omega^2\lambda^2)^2} \bigg(1 + \frac{z}{\lambda} -4\omega^2\lambda^2(1 - \frac{z}{\lambda})\bigg)\bigg] \\
& \times \exp\bigg[- i k z\frac{\alpha e^{-z/\lambda}}{(1+4\omega^2\lambda^2)^2} \bigg(1 + \frac{2\lambda}{z} + 4\omega^2\lambda^2\bigg) \bigg] h_{ab}^{(s)}(\omega) \,. \label{spesptsol}
\end{aligned}
\end{equation}
Furthermore if $\omega \lambda \gg 1$, the eq.\eqref{spesptsol} becomes
\begin{equation}
\begin{aligned}
h^{(o)}_{ab}(\omega) = &  \sqrt{\frac{G_o}{G_s}}\exp\bigg[\frac{\alpha}{8\omega^2\lambda^2} \bigg(1 - \frac{z}{\lambda}\bigg)e^{-z/\lambda} \bigg] \exp\bigg[- i k z\frac{\alpha e^{-z/\lambda}}{4\omega^2\lambda^2} \bigg] h_{ab}^{(s)}(\omega) \\
= & \sqrt{\frac{G_o}{G_s}}\exp\bigg[-\frac{\partial_z^2 \ln G_o}{8\lambda^2} \bigg] \exp\bigg[- i k z\frac{\alpha e^{-z/\lambda}}{4\omega^2\lambda^2} \bigg] h_{ab}^{(s)}(\omega) \,.
\end{aligned}
\end{equation}
\end{comment}
If $\phi = \phi(t)$, the new variable $h_{ij} = C_{ij}e^{ikz}H(t)$ obeys the relation from eq.\eqref{fereom}
\begin{align}
\partial_t^2 H + \partial_t\ln\phi \,\partial_t H  + k^2H= 0
\end{align}
\begin{comment}
This, in terms of eq.\eqref{lapsol} with $y(t) = -\frac{1}{2}\partial_t^2g(t)$, $g(t) = -\ln G$, also makes the GW having the following relation during propagation
\begin{equation}
\begin{aligned}
h^{(o)}_{ab}(\omega) = & \sqrt{\frac{G_o}{G_s}} \exp\bigg[-ik t\frac{\partial_t^2 \ln G_o}{4k^2} \bigg] h_{ab}^{(s)}(\omega) = \sqrt{\frac{G_o}{G_s}} \exp\bigg[-i\omega t\frac{\partial_t^2 \ln G_o}{4k^2} \bigg] h_{ab}^{(s)}(\omega)\,, \label{ferlapsol}
\end{aligned}
\end{equation}
We expect an dispersion relation $\omega = k + \varepsilon k^{(1)}$ both in eq.\eqref{fersptsol} and \eqref{ferlapsol} and the second order term is omit, actually the dispersion relations from the phase corrections are given by
\begin{align}
\omega = & k + \frac{\partial_z^2\ln G_o}{4k} \quad \bigg( = k + \frac{\nabla^2\ln G}{4k} ????? \bigg)\,, \\
\omega = & k - \frac{\partial_t^2\ln G_o}{4k}\,.
\end{align}
in eq.\eqref{fersptsol} and eq.\eqref{ferlapsol} respectively. The phase corrections in eq.\eqref{fersptsol} and \eqref{ferlapsol} are obvious depending on the second order derivative and proportional to $1/\omega$, while the amplitude correction depends on a factor $\sqrt{G_o/G_s}$.
\end{comment}

\section{Action method for Einstein-Hilbert action}
Using the \textit{xAct} module in Mathematica, we expand the Lagrangian of Einstein-Hilbert action up to second order
\begin{equation}
\begin{aligned}
\mathcal{L}_{\mathrm{GR}} = \sqrt{-g}R = & -\frac{1}{2}\partial_b h \partial^b h - 2\partial_a h^{ab}\partial_c h^c_b + 2\partial_b h\partial_c h^{bc} + h\partial_b\partial_c h^{bc} \\
& + 2h^{ab}(\partial_a\partial_b h-\partial_b\partial_c h^c_a - \partial_c\partial_b h^c_a + \square h_{ab})\\
& - h \square h - \partial_b h_{ac}\partial^c h^{ab} + \frac{3}{2}\partial_c h_{ab}\partial^c h^{ab}
\end{aligned}
\end{equation}
Using the transverse trace variable 
\begin{align}
\bar{h}_{ab} = h_{ab} - \frac{1}{2}\eta_{ab}h\,,~~h_{ab} = \bar{h}_{ab} - \frac{1}{2}\eta_{ab}\bar{h}
\end{align}
and the gauge condition
\begin{align}
\partial_a \bar{h}^{ab} = 0\,,~~\partial_ah^{ab} = \frac{1}{2} \partial^b h=-\frac{1}{2}\partial^b\bar{h}
\end{align}
it becomes 
\begin{equation}
\begin{aligned}
\sqrt{-g}R = & -\frac{1}{2}\partial_b h \partial^b h - 2\partial_a h^{ab}\partial_c h^c_b + 2\partial_b h\partial_c h^{bc} + h\partial_b\partial_c h^{bc} + 2h^{ab}(\partial_a\partial_b h-\partial_b\partial_c h^c_a - \partial_c\partial_b h^c_a + \square h_{ab})\\
& - h \square h - \partial_b h_{ac}\partial^c h^{ab} + \frac{3}{2}\partial_c h_{ab}\partial^c h^{ab} \\
= & -\frac{1}{2}\partial_bh\partial^b h - \frac{1}{2}\partial^bh\partial_bh + \partial_bh\partial^bh + \frac{1}{2} h\square h + 2h^{ab}\partial_a\partial_b h - 2h^{ab}\partial_b\partial_a h + h^{ab}\square h_{ab} - h\square h\\
& - \partial_b h_{ac}\partial^c h^{ab} + \frac{3}{2}\partial_c h_{ab}\partial^c h^{ab}\\
= & -\frac{1}{2} h\square h + h^{ab}\square h_{ab} - \partial_b h_{ac}\partial^c h^{ab} + \frac{3}{2}\partial_c h_{ab}\partial^c h^{ab} \\
= & -\frac{1}{2} \bar{h}\square \bar{h} + \bar{h}^{ab}\square \bar{h}_{ab} - \partial_b \bar{h}_{ac}\partial^c \bar{h}^{ab} - \frac{1}{4}\partial_b\bar{h} \partial^b\bar{h} + \frac{3}{2}\partial_c \bar{h}_{ab}\partial^c \bar{h}^{ab}
\end{aligned}
\end{equation}
where we used the relations
\begin{align}
h^{ab} \square h_{ab} = & 2(\bar{h}^{ab}-\frac{1}{2}\eta^{ab}\bar{h})\square (\bar{h}_{ab} - \frac{1}{2}\eta_{ab}\bar{h}) = 2\bar{h}^{ab} \square \bar{h}_{ab} - 2 \bar{h}\square \bar{h} + 2\bar{h}\square \bar{h} = \bar{h}^{ab} \square \bar{h}_{ab} \\
- h \square h = & -\bar{h}\square \bar{h}\\
- \partial_b h_{ac}\partial^c h^{ab} = & -\partial_b(\bar{h}_{ac} - \frac{1}{2}\eta_{ac}\bar{h})\partial^c (\bar{h}^{ab}-\frac{1}{2}\eta^{ab}\bar{h}) = - \partial_b \bar{h}_{ac}\partial^c \bar{h}^{ab} + \partial_b \bar{h}\partial_a\bar{h}^{ab} - \frac{1}{4}\partial_b\bar{h} \partial^b\bar{h} \\
= & - \partial_b \bar{h}_{ac}\partial^c \bar{h}^{ab} - \frac{1}{4}\partial_b \bar{h} \partial^b \bar{h} \\
\frac{3}{2}\partial_c h_{ab}\partial^c h^{ab}  = & \frac{3}{2}\partial_c (\bar{h}_{ab} - \frac{1}{2}\eta_{ab}\bar{h}) \partial^c (\bar{h}^{ab}-\frac{1}{2}\eta^{ab}\bar{h}) = \frac{3}{2}\partial_c \bar{h}_{ab}\partial^c \bar{h}^{ab} -\frac{3}{2}\partial_c \bar{h} \partial^c \bar{h} + \frac{3}{2}\partial_ch\partial^c \bar{h} = \frac{3}{2}\partial_c \bar{h}_{ab}\partial^c \bar{h}^{ab}
\end{align}
Similarly the action for the GW is
\begin{equation}
\begin{aligned}
S = & \frac{1}{16\pi G} \int d^4x\bigg[  -\frac{1}{2} \bar{h}\square \bar{h} + \bar{h}^{ab}\square \bar{h}_{ab} - \partial_b \bar{h}_{ac}\partial^c \bar{h}^{ab} - \frac{1}{4}\partial_b\bar{h} \partial^b\bar{h} + \frac{3}{2}\partial_c \bar{h}_{ab}\partial^c \bar{h}^{ab} \bigg] \\
= & \frac{1}{16\pi } \int d^4x ~ \phi\bigg[  -\frac{1}{2} \bar{h}\square \bar{h} + \bar{h}^{ab}\square \bar{h}_{ab} - \partial_b \bar{h}_{ac}\partial^c \bar{h}^{ab} - \frac{1}{4}\partial_b\bar{h} \partial^b\bar{h} + \frac{3}{2}\partial_c \bar{h}_{ab}\partial^c \bar{h}^{ab} \bigg] \\
\end{aligned}
\end{equation}
We work in TT gauge
\begin{equation}
\begin{aligned}
S = & \frac{1}{16\pi } \int d^4x ~ \phi\bigg[ \bar{h}^{ij}\square \bar{h}_{ij} - \partial_j \bar{h}_{ik}\partial^k \bar{h}^{ij} + \frac{3}{2}\partial_c \bar{h}_{ij}\partial^c \bar{h}^{ij} \bigg] \\
\end{aligned}
\end{equation}
and EOM becomes
\begin{equation}
\begin{aligned}
\phi\square \bar{h}_{ij} + \square (\phi \bar{h}_{ij}) + 2\partial^k(\phi\partial_j \bar{h}_{ik}) - \frac{3}{2}\partial^a (\phi \partial_a \bar{h}_{ij}) = 0
\end{aligned}
\end{equation}
or
\begin{equation}
\begin{aligned}
\square \bar{h}_{ij} + \bigg(\frac{2}{\phi}\square \phi \bigg)\bar{h}_{ij} + \partial^a \ln\phi \partial_a\bar{h}_{ij} + 4\partial^k \ln\phi\partial_j \bar{h}_{ik} = 0
\end{aligned}
\end{equation}
The first case assumed as $\phi = \phi(\vec{r})$, we obtain the differential equations of $\bar{h}_{ij}$
\begin{equation}
\begin{aligned}
\square \bar{h}_{ij} + \bigg(\frac{2}{\phi}\nabla^2 \phi\bigg) \bar{h}_{ij} + \partial^k \ln\phi \partial_k \bar{h}_{ij} + 4\partial^k \ln\phi\partial_j \bar{h}_{ik} = 0
\end{aligned}
\end{equation}
We then separate the variables as $\bar{h}_{ij} = H_{ij}e^{-i\omega t}$ so the spatial differential equations become
\begin{equation}
\begin{aligned}
\nabla^2H_{ij} + \bigg[ \omega^2 + \frac{2}{\phi}\nabla^2 \phi\bigg]H_{ij} + \partial^k \ln\phi \partial_k H_{ij} + 4\partial^k \ln\phi\partial_j H_{ik} = 0
\end{aligned}
\end{equation}
Because the GW travels along $z$-axis, the $H_{ij}$ is the function of $z$ only,
\begin{equation}
\begin{aligned}
    \partial_z^2 H_{ij} + \bigg[ \omega^2 + \frac{2}{\phi}\partial_z^2 \phi \bigg]H_{ij} + \partial_z \ln\phi \partial_z H_{ij} + 4\delta^z_j\partial^k \ln\phi\partial_z H_{ik} = 0
\end{aligned}
\end{equation}
In TT gauge, $H_{ij}$ furthermore only has $x,y$ components so the equations become
\begin{equation}
\begin{aligned}
    \partial_z^2 h+ \partial_z \ln\phi\,\partial_z h + \bigg[ \omega^2 + \frac{2}{\phi}\partial_z^2 \phi \bigg]h  = 0 \label{desspteq1}
\end{aligned}
\end{equation}
where $h$ represents $h_+$ or $h_\times$. Approximately, this system has a frequency shift $\frac{1}{\omega\phi}\partial_z^2\phi$.\begin{comment} This equation has the same form to the eq.\eqref{dessoleq} with $f(z) = \frac{2}{\phi}\nabla^2 \phi \approx 2\nabla^2\ln \phi$, $g(z) = \ln\phi$. Therefore the forward wave solution to eq.\eqref{desspteq1} is
\begin{equation}
\begin{aligned}
h^{(o)}_{ab}(\omega) = & \sqrt{\frac{G_o}{G_s}} \exp\bigg[iz \frac{ 4\nabla^2\ln \phi -\partial_z^2 \ln\phi}{4\omega} \bigg] h_{ab}^{(s)}(\omega) \\
= & \sqrt{\frac{G_o}{G_s}} \exp\bigg[-ikz \frac{ 4\nabla^2\ln G_o -\partial_z^2 \ln G_o}{4\omega^2} \bigg] h_{ab}^{(s)}(\omega) \,.
\label{ehsptsol}
\end{aligned}
\end{equation}
\textcolor{red}{The general form is ??}
\begin{equation}
h^{(o)}_{ab}(\omega) = \sqrt{\frac{G_o}{G_s}} \exp\bigg[ \frac{3\nabla^2 \ln G_o}{8\omega^2}\bigg] \exp\bigg[-i\frac{ 3\nabla^2 \ln G_o}{4\omega^2} \vec{k}\cdot\vec{r} \bigg] h_{ab}^{(s)}(\omega) ??
\end{equation}

Now we consider the specific form of gravitational constant as
\begin{equation}
G = G_s(1+\alpha(1+z/\lambda) e^{-z/\lambda})\,.
\end{equation}
Again using the two variable expansion method and matching the eq.\eqref{heq} with
\begin{align}
y(z) = & 2\nabla^2\ln\phi - \frac{1}{2}\partial_z^2\ln\phi = \frac{\alpha}{2\lambda^2}\bigg( 11 - \frac{3z}{\lambda} \bigg) e^{-z/\lambda}\,,\\
g(z) = & \ln\phi = -\ln G \approx - \alpha\bigg(1+\frac{z}{\lambda}\bigg)e^{-z/\lambda}\,,
\end{align}
the solution is 
\begin{equation}
\begin{aligned}
h = & A\cos(\omega z) + B\sin(\omega z) \\
& + \frac{\alpha e^{-z/\lambda}}{2(1+4\omega^2\lambda^2)^2}\bigg[ 5 - \frac{3z}{\lambda} + \bigg(44 - \frac{12z}{\lambda}\bigg) \lambda^2\omega^2 \bigg]\big[ A\cos(\omega z) + B\sin(\omega z) \big]  \\
& + \frac{\alpha e^{-z/\lambda}}{(1+4\omega^2\lambda^2)^2}\omega z \bigg[ 3 - \frac{2\lambda}{z} + \bigg(12 - \frac{16\lambda}{z} \bigg) \lambda^2\omega^2 \bigg] \big[A\sin(\omega z) - B\cos(\omega z)\big]
\end{aligned}
\end{equation}
The in-going wave condition $h = \mathcal{A}e^{i\omega z}$ requires $B = iA, A = \mathcal{A}$ and $h$ becomes
\begin{equation}
\begin{aligned}
h = & \mathcal{A}e^{i\omega z}\bigg\{ 1 + \frac{\alpha e^{-z/\lambda}}{2(1+4\omega^2\lambda^2)^2} \bigg[ 5 - \frac{3z}{\lambda} + \bigg(44 - \frac{12z}{\lambda}\bigg) \lambda^2\omega^2 \bigg] \\
&  - i \frac{\alpha e^{-z/\lambda}}{(1+4\omega^2\lambda^2)^2}\omega z \bigg[ 3 - \frac{2\lambda}{z} + \bigg(12 - \frac{16\lambda}{z} \bigg) \lambda^2\omega^2 \bigg] \bigg\} \\
= & \mathcal{A}e^{i\omega z} \exp\bigg\{ \frac{\alpha e^{-z/\lambda}}{2(1+4\omega^2\lambda^2)^2} \bigg[ 5 - \frac{3z}{\lambda} + \bigg(44 - \frac{12z}{\lambda}\bigg) \lambda^2\omega^2 \bigg] \bigg\} \\
& \times \exp\bigg\{ i \frac{\alpha e^{-z/\lambda}}{(1+4\omega^2\lambda^2)^2}\omega\lambda \bigg[ 2 -\frac{3z}{\lambda} + \bigg(16 - \frac{12z}{\lambda} \bigg) \lambda^2\omega^2 \bigg] \bigg\} \,.
\end{aligned}
\end{equation}
Therefore the GW is
\begin{equation}
\begin{aligned}
h^{(o)}_{ab}(\omega) = &  \sqrt{\frac{G_o}{G_s}} \exp\bigg\{ \frac{\alpha e^{-z/\lambda}}{2(1+4\omega^2\lambda^2)^2} \bigg[ 5 - \frac{3z}{\lambda} + \bigg(44 - \frac{12z}{\lambda}\bigg) \lambda^2\omega^2 \bigg] \bigg\} \\
& \times \exp\bigg\{ i \frac{\alpha e^{-z/\lambda}}{(1+4\omega^2\lambda^2)^2}\omega\lambda \bigg[ 2 -\frac{3z}{\lambda} + \bigg(16 - \frac{12z}{\lambda} \bigg) \lambda^2\omega^2 \bigg] \bigg\} h^{(s)}_{ab}(\omega) \,. \label{ehspesptsol}
\end{aligned}
\end{equation}
Furthermore if $\omega \lambda \gg 1$, the eq.\eqref{ehspesptsol} becomes
\begin{equation}
\begin{aligned}
h^{(o)}_{ab}(\omega) = & \sqrt{\frac{G_o}{G_s}} \exp\bigg[ \frac{\alpha e^{-z/\lambda}}{8\omega^2\lambda^2} \bigg(11 - \frac{3z}{\lambda}\bigg)  \bigg] \exp\bigg[ i \frac{\alpha e^{-z/\lambda}}{\omega\lambda} \bigg(1 - \frac{3z}{4\lambda} \bigg) \bigg] h_{ab}^{(s)}(\omega) \\
= & \sqrt{\frac{G_o}{G_s}} \exp\bigg[ -\frac{1}{4\omega^2} \bigg(2\nabla^2\ln G - \frac{1}{2}\partial_z^2\ln G\bigg)  \bigg] \exp\bigg[ i \frac{\alpha e^{-z/\lambda}}{\omega\lambda} \bigg(1 - \frac{3z}{4\lambda} \bigg) \bigg] h_{ab}^{(s)}(\omega) \,.
\end{aligned}
\end{equation}
\end{comment}


If the dependence becomes $\phi = \phi(t)$, the EOM becomes
\begin{equation}
\begin{aligned}
\frac{1}{2}\square \bar{h}_{ij} + \frac{\bar{h}_{ij}}{\phi}\square \phi + \frac{1}{2} \partial^a \ln\phi \partial_a\bar{h}_{ij} + 2\partial^k \ln\phi\partial_j \bar{h}_{ik} = 0
\end{aligned}
\end{equation}
or after variable separation $\bar{h}_{ij} = C_{ij}e^{ikz}H(t)$
\begin{equation}
\begin{aligned}
\partial_t^2H + \partial_t\ln\phi\,\partial_tH + \left[k^2 + \frac{2}{\phi}\partial_t^2 \phi\right] H = 0
\end{aligned}
\end{equation}
\begin{comment}
This, in terms of eq.\eqref{lapsol} with $y(t) = 2\partial_t^2\phi - \frac{1}{2}\partial_t^2\ln\phi = \frac{3}{2}\partial_t^2\ln\phi = -\frac{3}{2}\partial_t^2\ln G$, $g(t) = -\ln G$, also makes the forward GW having the following relation during propagation
\begin{equation}
\begin{aligned}
h^{(o)}_{ab}(\omega) = & \sqrt{\frac{G_o}{G_s}} \exp\bigg[ik t\frac{3\partial_t^2 \ln G_o}{4k^2} \bigg] h_{ab}^{(s)}(\omega) = \sqrt{\frac{G_o}{G_s}} \exp\bigg[i\omega t\frac{3\partial_t^2 \ln G_o}{4k^2} \bigg] h_{ab}^{(s)}(\omega)\,. \label{ehlapsol}
\end{aligned}
\end{equation}
The dispersion relations from the phase corrections are given by
\begin{align}
\omega = & k - \frac{ 4\nabla^2\ln G_o -\partial_z^2 \ln G_o}{4k} \quad \bigg( = k + \frac{ 3\nabla^2 \ln G_o}{4k}????? \bigg) \,, \\
\omega = & k + \frac{ 3\partial_t^2 \ln G_o}{4k} \,.
\end{align}
in eq.\eqref{ehsptsol} and eq.\eqref{ehlapsol} respectively. These dispersion relation gives rise to the derivate from light speed by
\begin{align}
|\nabla^2\ln G| < \frac{16\pi^2}{3} f^2 \Delta c\,, \\
|\partial_t^2\ln G| < \frac{16\pi^2}{3} f^2 \Delta c\,.
\end{align}
\end{comment}

\section{Appendix A: Solving wave equation}
\label{sec:appa}
In this appendix, we solve the wave equation. By expressing $h_{ij}$ as a Fourier transform, the
wave equation for each Fourier amplitude becomes the following general linear form
\begin{equation}
    \frac{d^2}{d z^2}H(z)+2p(z)\frac{d}{d z}H(z)+\left[\omega^2+q(z)\right]H(z)=0.\label{dessoleq}
\end{equation}

For any $H(z)$, there are real function $A(z)$ and $\Phi(z)$ which satisfy
\begin{equation}\label{HAphi}
    H=Ae^{i\Phi}.
\end{equation}
We plug eq.\eqref{HAphi} into eq.\eqref{dessoleq} and then divide both sides of the equation by $e^{i\Phi}$. From the real and imaginary part of the result, we achieve two equations,
\begin{equation}\label{rpart}
    \frac{d^2 A}{d z^2}+2p\frac{d A}{d z}+\left[\omega^2\left(1-\frac{k^2}{\omega^2}\right)+q\right]A=0,
\end{equation}
\begin{equation}\label{ipart}
    2\frac{d A}{d z}k+A\frac{d k}{d z}+2pAk=0,
\end{equation}
where $k=\frac{d \Phi}{d z}$. It is no matter for us to suppose that $A>0$ and $k>0$, and then
\begin{equation}
    2\frac{1}{A}\frac{d A}{d z}+\frac{1}{k}\frac{d k}{d z}+2p=0,
\end{equation}
therefore
\begin{equation}\label{Apk}
    A\propto e^{-\int p\,dz}k^{-1/2}.
\end{equation}
We plug eq.\eqref{Apk} into eq.\eqref{rpart} and then achieve
\begin{equation}
    \frac{d^2 K}{d z^2}-\left(\frac{1}{\Gamma}\frac{d^2\Gamma}{d z^2}-q\right)K+\omega^2K(1-K^{-4})=0,
\end{equation}
where $\Gamma=e^{\int p \,dz}$ and $K=(k/\omega)^{-1/2}$.

We suppose that $K=1+\Delta K$, where we speculate that $\Delta K$ is a small amount, and then
\begin{comment}
\begin{equation}
    \frac{d^2 \Delta K}{d z^2}-\left(\frac{1}{\Gamma}\frac{d^2\Gamma}{d z^2}-q\right)(1+\Delta K)+\omega^2(1+\Delta K)\sum_{n=1}^{\infty}(-1)^{n+1}\frac{(n+3)!}{n!3!}{(\Delta K)}^n=0.
\end{equation}
\end{comment}
\begin{comment}
\begin{equation}
    \frac{d^2 \Delta K}{d z^2}-\left(\frac{1}{\Gamma}\frac{d^2\Gamma}{d z^2}-q\right)(1+\Delta K)+\omega^2(1+\Delta K)\sum_{n=0}^{\infty}(-1)^{n+1}C^3_{n+3}{(\Delta K)}^n=0,
\end{equation}
\end{comment}
\begin{comment}
\begin{equation}
    \frac{d^2 \Delta K}{d z^2}-\left(\frac{1}{\Gamma}\frac{d^2\Gamma}{d z^2}-q\right)(1+\Delta K)+\omega^2[4\Delta K+\sum_{n=1}^{\infty}((-1)^{n+1}(C^3_{n+3}-C^3_{n+4})){(\Delta K)}^{n+1}]=0.
\end{equation}
\end{comment}
\begin{equation}
    \frac{d^2 \Delta K}{d z^2}-\Xi(1+\Delta K)+\omega^2\left[4\Delta K+\sum_{n=2}^{\infty}(-1)^{n+1}C^2_{n+2}{\Delta K}^n\right]=0,
\end{equation}
where $\Xi=\frac{1}{\Gamma}\frac{d^2\Gamma}{d z^2}-q$. 
If we neglect the higher order terms, then
\begin{equation}\label{DKoscillation}
    \frac{d^2 \Delta K}{d z^2}+(2\omega)^2\Delta K=\Xi.
\end{equation}
Solved eq.\eqref{DKoscillation} by using constant variation method, we achieve
\begin{equation}
    \Delta K=\left[\Delta K|_{z=0}-\int_{0}^z\frac{\Xi\sin(2\omega z)}{2\omega}\,d z\right]\cos(2\omega z)+\left[\frac{1}{2\omega}\frac{d \Delta K}{d z}|_{z=0}+\int_{0}^z\frac{\Xi\cos(2\omega z)}{2\omega}\,d z\right]\sin(2\omega z).
\end{equation}
We only need some particular solutions, because eq.\eqref{dessoleq} is linear. We make $\Delta K|_{z=0}=0$ and $\frac{d \Delta K}{d z}|_{z=0}=0$, and then
\begin{equation}
    \Delta K=-\left[\int_{0}^z\frac{\Xi\sin(2\omega z)}{2\omega}\,d z\right]\cos(2\omega z)+\left[\int_{0}^z\frac{\Xi\cos(2\omega z)}{2\omega}\,d z\right]\sin(2\omega z),
\end{equation}
which means
\begin{equation}
    \Delta K=\frac{\Xi}{(2\omega)^2}[1-\cos(2\omega z)]-\left[\int_{0}^z\frac{\frac{d\Xi}{d z}\cos(2\omega z)}{(2\omega)^2}\,d z\right]\cos(2\omega z)-\left[\int_{0}^z\frac{\frac{d\Xi}{d z}\sin(2\omega z)}{(2\omega)^2}\,d z\right]\sin(2\omega z).
\end{equation}
There is
\begin{equation}
    \left\lvert \left[\int_{0}^z\frac{\frac{d\Xi}{d z}\cos(2\omega z)}{(2\omega)^2}\,d z\right]\cos(2\omega z)+\left[\int_{0}^z\frac{\frac{d\Xi}{d z}\sin(2\omega z)}{(2\omega)^2}\,d z\right]\sin(2\omega z)\right\rvert \leq 2\int_{0}^z\frac{\left\lvert \frac{d\Xi}{d z}\right\rvert }{(2\omega)^2},
\end{equation}
because $\left|\cos(2\omega z)\right|\leq1$ and $\left|\sin(2\omega z)\right|\leq1$. Consequently, if $\left|\Xi(z)\right|\ll(2\omega)^2$ and $\Xi(z)$ has only a few monotonic intervals, we can be sure that this approximate solution is truly a small amount. Finally, obviously we can approximate $k$ by $k=\omega(1+\Delta K)^{-2}\approx \omega(1-2\Delta K)$.

\begin{comment}
\begin{equation}
    H=C_+e^{-pz}e^{+i\omega\sqrt{1-(\frac{p}{\omega})^2}z}+C_-e^{-pz}e^{-i\omega\sqrt{1-(\frac{p}{\omega})^2}z}
\end{equation}
\begin{equation}
    H|_{z=0}=C_++C_-=1
\end{equation}
\begin{equation}
    \frac{d H}{d z}|_{z=0}=C_+(-p+i\omega\sqrt{1-(\frac{p}{\omega})^2})+C_-(-p-i\omega\sqrt{1-(\frac{p}{\omega})^2})=i\omega
\end{equation}
\begin{equation}
    C_\pm=\frac{1}{2}\pm\frac{1-\frac{p}{\omega}i}{2\sqrt{1-(\frac{p}{\omega})^2}}
\end{equation}
\begin{equation}
    H=(\frac{1}{2}+\frac{1-\frac{p}{\omega}i}{2\sqrt{1-(\frac{p}{\omega})^2}})e^{-pz}(\cos\omega\sqrt{1-(\frac{p}{\omega})^2}z+i\sin\omega\sqrt{1-(\frac{p}{\omega})^2}z)+(\frac{1}{2}-\frac{1-\frac{p}{\omega}i}{2\sqrt{1-(\frac{p}{\omega})^2}})e^{-pz}(\cos\omega\sqrt{1-(\frac{p}{\omega})^2}z-i\sin\omega\sqrt{1-(\frac{p}{\omega})^2}z)
\end{equation}
\begin{equation}
    H=[e^{-pz}\cos\omega\sqrt{1-(\frac{p}{\omega})^2}z+\frac{\frac{p}{\omega}}{\sqrt{1-(\frac{p}{\omega})^2}}e^{-pz}\sin\omega\sqrt{1-(\frac{p}{\omega})^2}z]+i[\frac{1}{\sqrt{1-(\frac{p}{\omega})^2}}e^{-pz}\sin\omega\sqrt{1-(\frac{p}{\omega})^2}z]
\end{equation}
\end{comment}

\begin{comment}
Finally, we obtain the general solution of eq.\eqref{dessoleq},
\begin{equation}
    H(z)=C_+e^{-\int p dz}Ke^{+i\int \omega K^{-2} dz}+C_-e^{-\int p dz}Ke^{-i\int \omega K^{-2} dz}.
\end{equation}
If $H(z)$ is real function, it will be
\begin{equation}
    H(z)=C_\text{c}e^{-\int p dz}K\cos(\int \omega K^{-2} dz)+C_\text{s}e^{-\int p dz}K\sin(\int \omega K^{-2} dz).
\end{equation}
\end{comment}

\begin{comment}
Finally, obviously we can approximate $k$ by $k=\omega(1+\Delta K)^{-2}\approx \omega(1-2\Delta K)$.

\begin{equation}
    H=C_+e^{-pz}e^{+i\omega\sqrt{1-(\frac{p}{\omega})^2}z}+C_-e^{-pz}e^{-i\omega\sqrt{1-(\frac{p}{\omega})^2}z}
\end{equation}
\end{comment}

\begin{comment}
\begin{equation}
    \begin{aligned}
    \partial_z^2 H  + \alpha\partial_z g(z) \partial_z H + \bigg[ \omega^2 + \alpha f(z) \bigg]H= 0\,.\label{dessoleq}
    \end{aligned}
    \end{equation}
\begin{comment}
\begin{equation}
\begin{aligned}
\partial_z^2 H  + \alpha\partial_z g(z) \partial_z H + \bigg[ \omega^2 + \alpha f(z) \bigg]H= 0\,.\label{dessoleq}
\end{aligned}
\end{equation}
We define the variable $h(z)$ through
\begin{align}
H(z) = \exp\bigg[-\frac{1}{2}\alpha\int \partial_zg(z)dz  \bigg]h(z) = \exp\bigg\{ \frac{1}{2}\alpha [g(z_1)-g(z)] \bigg\} h(z) 
\end{align}
and we have the new equation for $h(z)$
\begin{equation}
\begin{aligned}
\ddot{h}(z) + \bigg\{ \omega^2 + \alpha[f(z) - \frac{1}{2}\partial_z^2g(z)] + \frac{1}{4}\alpha^2[\partial_zg(z)]^2 \bigg\} h(z) = 0
\end{aligned}
\end{equation}
Since we only want to see the leading order, we
rewrite the equation as
\begin{equation}
\begin{aligned}
\ddot{h}(z) + \big[ \omega^2 + \alpha y(z) \big] h(z) = 0\,,~~y(z) = f(z) - \frac{1}{2}\partial_z^2g(z)\,.  \label{heq}
\end{aligned}
\end{equation}
We solve the equation \eqref{heq} by the two variable expansion method treating the $h(z)$ as a function of the two independent variables $(\xi,\eta)$ so that the derivative respect to $z$ become 
\begin{equation}
\frac{d}{dz} = \frac{\partial}{\partial \xi} + \alpha \frac{\partial}{\partial \eta}
\end{equation}
We expand $h$ and the terms in eq.\eqref{heq}
\begin{align*}
& h(\xi,\eta) = h_0(\xi,\eta) + \alpha h_1(\xi,\eta) + \cdots \\
& \partial_z^2 h(\xi,\eta) = \partial_\xi^2 h_0(\xi,\eta) + 2\alpha \partial_\xi\partial_\eta h_0(\xi,\eta) + \alpha \partial_\xi^2 h_1(\xi,\eta) + \alpha^2\frac{\partial^2}{\partial\eta^2}h_0 + 2\alpha^2\partial_\xi\partial_\eta h_1 + \alpha^2\partial_\xi^2 h_2 + \cdots \\
& (\omega^2 + \alpha y(z)) h(\xi,\eta) = \omega^2 h_0(\xi,\eta) + \alpha\omega^2 h_1(\xi,\eta) + \alpha y(\xi) h_0(\xi,\eta) + \alpha^2y(\xi)h_1(\xi,\eta) + \alpha^2\omega^2 h_2(\xi,\eta) + \cdots \,,
\end{align*}
which gives rise to three differential equations
\begin{align}
& (\partial_\xi^2 + \omega^2) h_0(\xi,\eta) = 0 \label{diff1}\,, \\
& (\partial_\xi^2 + \omega^2) h_1(\xi,\eta) = -2\partial_\xi\partial_\eta h_0(\xi,\eta) - y(\xi)h_0(\xi,\eta)\,. \label{diff2} \\
& (\partial_\xi^2 + \omega^2) h_2(\xi,\eta) = -2\partial_\xi\partial_\eta h_1(\xi,\eta) - \partial_\eta^2 h_0(\xi,\eta) - y(\xi)h_1(\xi,\eta)
\end{align}
The equation \eqref{diff1} gives $h_0$ in the form of
\begin{equation}
\begin{aligned}
h_0 = A(\eta)\cos(\omega\xi) + B(\eta)\sin(\omega\xi) \,, \label{h0}
\end{aligned}
\end{equation}
and putting this into eq.\eqref{diff2} gives
\begin{equation}
\begin{aligned}
(\partial_\xi^2 + \omega^2) h_1(\xi,\eta) = & 2\omega(A'\sin(\omega\xi)-B'\cos(\omega\xi)) - y(\xi)(A\cos(\omega\xi) + B\sin(\omega\xi)) \\
= & (2\omega A' - y(\xi)B)\sin(\omega\xi) - (2\omega B'+ y(\xi)A)\cos(\omega\xi) \label{diff2sol}
\end{aligned}
\end{equation}
Referring to the method to solve Mathieu equation, the RHS in eq.\eqref{diff2sol} vanishies to solve $A$ and $B$, and we obtain
\begin{equation}
\begin{aligned}
\frac{dA}{d\eta} = \frac{y(\xi)}{2\omega}B\,, ~~ \frac{dB}{d\eta} = - \frac{y(\xi)}{2\omega}A\,,
\end{aligned}
\end{equation}
However the two differential equations of $A$ and $B$ also depend on $y(\xi)$ or variable $\xi$, which breaks the condition that only relies on $\eta$. (If $y(\xi)$ is a constant, it is easy to solve eq.\eqref{heq} by modifying $\tilde{\omega}^2 = \omega^2 + \alpha y$.) Therefore $A$ and $B$ must be constants (actually $h_0$ now is the initial unperturbed solution) and the equation \eqref{diff2sol} have the form of
\begin{equation}
\begin{aligned}
(\partial_\xi^2 + \omega^2) h_1(\xi,\eta) = - y (\xi)B\sin(\omega\xi) -  y(\xi)A \cos(\omega\xi) \label{diff2sol2}
\end{aligned}
\end{equation}
We assume the ansatz
\begin{equation}
h_1 = C(\eta)D(\xi)\,,
\end{equation}
and substituting it into eq.\eqref{diff2sol2} leads to the differential equation
\begin{equation}
\begin{aligned}
C(\eta)(\partial_\xi^2 + \omega^2) D(\xi) = -Ay(\xi)\cos(\omega\xi) -B  y (\xi)\sin(\omega\xi)\,. \label{diff2sol3}
\end{aligned}
\end{equation}
Obviously $C(\eta) = \mathrm{const}$. The solution to eq.\eqref{diff2sol3} also depends on the concrete form of $y(\xi)$. We treat the $y(\xi)$ nearly as a constant here and $h_1$ is a resonant solution
\begin{align}
h_1 = CD(\xi) = \frac{-Ay(\xi)\omega\xi\sin(\omega\xi) + By(\xi)\omega\xi\cos(\omega\xi)}{2\omega^2}\,. \label{sol}
\end{align}
If the constant condition is not satisfied, one should 
exactly solve the equation
\begin{equation}
\begin{aligned}
\frac{d^2}{d^2\xi}h_1(\xi) + \omega^2 h_1(\xi) = -Ay(\xi)\cos(\omega\xi) -B  y (\xi)\sin(\omega\xi)\,, \label{desfirorddiffeq}
\end{aligned}
\end{equation}
but for a very large $\xi$($=z$), the resonant result, namely eq.\eqref{sol}, \textcolor{red}{must be dominated ?????}. Finally combining with eq.\eqref{h0} and \eqref{sol} the solution to $h$ is
\begin{equation}
h = h_0 + \alpha h_1 = A\cos(\omega z) + B\sin(\omega z) + \alpha \frac{-Ay(z)\omega z\sin(\omega z) + By(z)\omega z\cos(\omega z)}{2\omega^2}\,.
\end{equation}
One can check that this solution is equivalent to 
\begin{align}
h = A \cos[\sqrt{\omega^2 + \alpha y(z)} z] + B \sin[\sqrt{\omega^2 + \alpha y(z)} z]\,,
\end{align}
for a small $\alpha$ expansion to leading order. This implies that the approximate solution is based on the requirement that $y(z)$ could be regarded as a constant or varies slowly during propagation. $H$ is
\begin{equation}
\begin{aligned}
H(z) = & \exp\bigg\{ \frac{1}{2}\alpha [g(z_1)-g(z)] \bigg\} h(z) \\
= & \exp\bigg\{ \frac{1}{2}\alpha [g(z_1)-g(z)] \bigg\} \times \bigg[A \cos(\sqrt{\omega^2 + \alpha y(z)} z) + B \sin(\sqrt{\omega^2 + \alpha y(z)} z) \bigg] \\
= & \exp\bigg\{ \frac{1}{2}\alpha [g(z_1)-g(z)] \bigg\} \times \bigg\{\tilde{A} \exp\bigg[i\sqrt{\omega^2 + \alpha y(z)} z\bigg] + \tilde{B} \exp\bigg[-i\sqrt{\omega^2 + \alpha y(z)} z\bigg] \bigg\} \,, \label{finalsol}
\end{aligned}
\end{equation}
where the coefficients $\tilde{A}$ and $\tilde{B}$ are determined by initial conditions or boundary conditions. 

1) If $z$ is the spatial coordinate and the initial condition is a in-going wave, at the source position, we can derive $A$ and $B$ for $\alpha=0$ via
\begin{equation}
h_0 = \mathcal{A} e^{i\omega z}\,,
\end{equation}
giving
\begin{equation}
\tilde{A} = \mathcal{A}\,,~~\tilde{B} = 0\,.
\end{equation}
Substituting them into eq.\eqref{finalsol}, we obtain
\begin{equation}
\begin{aligned}
H = & \mathcal{A} \exp\bigg[i\sqrt{\omega^2 + \alpha y(z)} z\bigg] \exp\bigg\{ \frac{1}{2}\alpha [g(z_1)-g(z)] \bigg\} \\
= & \mathcal{A} \exp\bigg[i\omega z + i z\frac{\alpha y(z)}{2\omega} \bigg] \exp\bigg\{ \frac{1}{2}\alpha [g(z_1)-g(z)] \bigg\} \,.
\label{sptsol}
\end{aligned}
\end{equation}

2) If $z$ is the time coordinate and the initial condition is a in-going wave, at the source position, we can derive $A$ and $B$ via
\begin{equation}
h_0 = \mathcal{A} e^{-i\omega t}\,,
\end{equation}
where we replace symbol $z$ as $t$. This condition gives 
\begin{align}
\tilde{B} = \mathcal{A}\,,~~\tilde{A} = 0\,.
\end{align}
Substituting them into eq.\eqref{finalsol}, we obtain
\begin{equation}
\begin{aligned}
H = & \mathcal{A} \exp\bigg[-i\sqrt{\omega^2 + \alpha y(t)} t\bigg] \exp\bigg\{ \frac{1}{2}\alpha [g(t_1)-g(t)] \bigg\} \\
= & \mathcal{A} \exp\bigg[-i\omega t - i t\frac{\alpha y(t)}{2\omega} \bigg] \exp\bigg\{ \frac{1}{2}\alpha [g(t_1)-g(t)] \bigg\} \,.
\label{lapsol}
\end{aligned}
\end{equation}
\end{comment}









\bibliography{reference}

\end{document}


