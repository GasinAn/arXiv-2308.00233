\documentclass[
    jor,
    amsmath,amssymb,preprint,
    superscriptaddress,
    %reprint,
    %author-year,
    %author-numerical,
]{revtex4-2}

\usepackage{subfig}
\usepackage{color}
\usepackage{graphicx}% Include figure files
\usepackage{dcolumn}% Align table columns on decimal point
\usepackage{bm}% Bold math
%\usepackage[mathlines]{lineno}% Enable numbering of text and display math
%\linenumbers\relax % Commence numbering lines
\usepackage{hyperref}
\usepackage{verbatim}

\def\d{\mathrm{d}}
\def\p{\partial}

\begin{document}

\preprint{AIP/123-QED}

\title[Probe Varying Gravitational Constant via GW]{Probe Varying Gravitational Constant via GW}% Force line breaks with \\

\author{Bing Sun}
\affiliation{
CAS Key Laboratory of Theoretical Physics, Institute of Theoretical Physics, Chinese Academy of Sciences, Beijing 100190, China
}
\author{Jiachen An}
\affiliation{
Institute for Frontiers in Astronomy and Astrophysics,
Beijing Normal University, Beijing 102206, China
}
\affiliation{
Department of Astronomy, Beijing Normal University, Beijing 102206, China
}

\date{\today}% It is always \today, today,
             % but any date may be explicitly specified

\begin{abstract}
In this note, we give the effect of varying gravitational constant $G$ on  gravitational waves (GWs) during propagations. We separately make $G$ in Fierz-Pauli action and Einstein-Hilbert action vary, and then we maintain two modified GW equations with the same form. We then solve the equations for the situation in our interest, and our solution can be easily applied to gravitational constant in Yukawa form when it has a minor correction and a slow variation. However, finally we find that it is impossible for us to measure the spatial variation of $G$ during GW propagations by only GW observations, and it is discussed in the final Appendix.
\end{abstract}

\keywords{gravitational constant, gravitational waves}

\maketitle

\section{Action Method for Fierz-Pauli Action}

The Fierz-Pauli action for the field variable $h_{ab}$ is given by~\cite{Roy2020}
\begin{equation}
\begin{aligned}
S = & \,\frac{1}{64 \pi G} \int d^4 x\left(-\partial_a h_{b c} \partial^a h^{b c}+\partial_a h_b^b \partial^a h_c^c-2 \partial_a h_c^a \partial^c h_b^b+2 \partial_a h_c^a \partial_b h^{b c}\right).
\end{aligned}
\end{equation}
We set gravitational constant $G$ as a function in a form $\phi^{-1}$ so that the action becomes
\begin{equation}
\begin{aligned}
S = & \,\frac{1}{64 \pi} \int d^4 x~\phi\left(-\partial_a h_{b c} \partial^a h^{b c}+\partial_a h_b^b \partial^a h_c^c-2 \partial_a h_c^a \partial^c h_b^b+2 \partial_a h_c^a \partial_b h^{b c}\right). \label{origact}
\end{aligned}
\end{equation}
In the Lorentz gauge of transverse trace variable $\bar{h}_{ab} = h_{ab} - \frac{1}{2}\eta_{ab}h$, $\partial^b \bar{h}_{ab} = 0$, the action \eqref{origact} has a simple form
\begin{equation}
\begin{aligned}
S = & \,\frac{1}{64 \pi} \int d^4 x~\phi \left( -\partial_a\bar{h}_{bc}\partial^a\bar{h}^{bc} + \frac{1}{2}\partial_c\bar{h}\partial^c\bar{h} \right).
\end{aligned}
\end{equation}
The equation of motion (EOM) for this action becomes
\begin{align}
\square h_{ab} + \partial_ch_{ab}\partial^c\ln\phi = 0. \label{fereom}
\end{align}
where $h_{ab} = \bar{h}_{ab} - \frac{1}{2} \eta_{ab}\bar{h}$. We separate the variable as $h_{ab} = H_{ab}e^{-i\omega t}$, and then
\begin{equation}
\begin{aligned}
\nabla^2 H_{ab} + \omega^2 H_{ab} + i\omega \partial_t\ln \phi H_{ab} + \delta^{ij} \partial_j\ln \phi \partial_iH_{ab} = 0.
\end{aligned}
\end{equation}
If the GW travels along $z$-axis in a transverse-traceless (TT) gauge, $H_{ij} = H_{ij}(z)$ and the equation becomes
\begin{align}
\partial_z^2 H_{ij} + \omega^2 H_{ij} + i\omega \partial_t\ln \phi H_{ij} + \partial_z\ln \phi \partial_zH_{ij} = 0. \label{ferorg}
\end{align}
Moreover, if $\phi = \phi(\vec{r})$, the expression eq.\eqref{ferorg} is solvable
\begin{equation}\label{emz}
\begin{aligned}
\partial_z^2 H_{ij} + \partial_z \ln\phi\,\partial_z H_{ij} + \omega^2 H_{ij} = 0.
\end{aligned}
\end{equation}
In addition, if $\phi \neq \phi(\vec{r})$ but $\phi = \phi(t)$, the new variable $h_{ij} = C_{ij}e^{ikz}H(t)$ obeys the relation from eq.\eqref{fereom}, and then
\begin{equation}\label{emt}
\begin{aligned}
\partial_t^2 H + \partial_t\ln\phi \,\partial_t H  + k^2H= 0,
\end{aligned}
\end{equation}
which has a similar form like eq.\eqref{emz}.

Obviously, eq.\eqref{emz} has the form like eq.\eqref{dessoleq} in Appendix \ref{sec:appa}. Now we use definitions and methods in Appendix \ref{sec:appa}. For eq.\eqref{emz},
\begin{equation}
    p=\frac{1}{2}\partial_z \ln\phi,\quad q=0,
\end{equation}
therefore $\Gamma\propto\phi^{1/2}\propto G^{-1/2}$ and $\Xi=-\frac{1}{4}\phi^{-2}(\partial_z\phi)^2+\frac{1}{2}\phi^{-1}\partial_z^2\phi=\frac{3}{4}G^{-2}(\partial_zG)^2-\frac{1}{2}G^{-1}\partial_z^2G$. If $G$ takes the Yukawa form~\cite{Adelberger2003}
\begin{equation}\label{Yukawa}
    G=G_\infty[1+\alpha(1+z/\lambda)e^{-z/\lambda}],
\end{equation}
there will be
\begin{equation}
    \Xi=\frac{1}{\lambda^2}\frac{+[\alpha(z/\lambda)e^{-z/\lambda}]^2+2[\alpha e^{-z/\lambda}]^2-2[\alpha(z/\lambda)e^{-z/\lambda}]+2[\alpha e^{-z/\lambda}]}{4[\alpha(z/\lambda+1)e^{-z/\lambda}+1]^2}.
\end{equation}
Moreover, if $\left\lvert\frac{ \Xi}{\omega^2}\right\rvert \ll 1$ and $1/\lambda\ll\omega/c$, which means that $G$ has has minor correction and slow variation, there will be $K\approx1+\frac{1}{4}\frac{\Xi}{\omega^2}\approx1$ and $\omega\int K^{-2}\,d z\approx\omega\int (1+\frac{1}{4}\frac{\Xi}{\omega^2})^{-2}\,d z\approx\omega\int (1-\frac{1}{2}\frac{\Xi}{\omega^2})\,d z$. Consequently, the amplitude of GW, which is proportional to $\Gamma^{-1}K$, will be approximately proportional to $G^{1/2}$, and the phase of GW, $\omega\int K^{-2}\,d z$, will have an approximate modification $-\omega\int \frac{1}{2}\frac{\Xi}{\omega^2}\,d z$.

\section{Action Method for Einstein-Hilbert Action}

Using the \textit{xAct} module in Mathematica, we expand the Lagrangian of Einstein-Hilbert action up to second order
\begin{equation}
\begin{aligned}
\mathcal{L}_{\mathrm{GR}} = \sqrt{-g}R = & -\frac{1}{2}\partial_b h \partial^b h - 2\partial_a h^{ab}\partial_c h^c_b + 2\partial_b h\partial_c h^{bc} + h\partial_b\partial_c h^{bc} \\
& + 2h^{ab}(\partial_a\partial_b h-\partial_b\partial_c h^c_a - \partial_c\partial_b h^c_a + \square h_{ab})\\
& - h \square h - \partial_b h_{ac}\partial^c h^{ab} + \frac{3}{2}\partial_c h_{ab}\partial^c h^{ab}.
\end{aligned}
\end{equation}
Using the transverse trace variable 
\begin{align}
\bar{h}_{ab} = h_{ab} - \frac{1}{2}\eta_{ab}h\,,~~h_{ab} = \bar{h}_{ab} - \frac{1}{2}\eta_{ab}\bar{h},
\end{align}
and the gauge condition
\begin{align}
\partial_a \bar{h}^{ab} = 0\,,~~\partial_ah^{ab} = \frac{1}{2} \partial^b h=-\frac{1}{2}\partial^b\bar{h},
\end{align}
it becomes 
\begin{equation}
\begin{aligned}
\sqrt{-g}R = & -\frac{1}{2}\partial_b h \partial^b h - 2\partial_a h^{ab}\partial_c h^c_b + 2\partial_b h\partial_c h^{bc} + h\partial_b\partial_c h^{bc} \\
& + 2h^{ab}(\partial_a\partial_b h-\partial_b\partial_c h^c_a - \partial_c\partial_b h^c_a + \square h_{ab})\\
& - h \square h - \partial_b h_{ac}\partial^c h^{ab} + \frac{3}{2}\partial_c h_{ab}\partial^c h^{ab} \\
= & -\frac{1}{2}\partial_bh\partial^b h - \frac{1}{2}\partial^bh\partial_bh + \partial_bh\partial^bh + \frac{1}{2} h\square h \\
& + 2h^{ab}\partial_a\partial_b h - 2h^{ab}\partial_b\partial_a h + h^{ab}\square h_{ab} - h\square h\\
& - \partial_b h_{ac}\partial^c h^{ab} + \frac{3}{2}\partial_c h_{ab}\partial^c h^{ab}\\
= & -\frac{1}{2} h\square h + h^{ab}\square h_{ab} - \partial_b h_{ac}\partial^c h^{ab} + \frac{3}{2}\partial_c h_{ab}\partial^c h^{ab} \\
= & -\frac{1}{2} \bar{h}\square \bar{h} + \bar{h}^{ab}\square \bar{h}_{ab} - \partial_b \bar{h}_{ac}\partial^c \bar{h}^{ab} - \frac{1}{4}\partial_b\bar{h} \partial^b\bar{h} + \frac{3}{2}\partial_c \bar{h}_{ab}\partial^c \bar{h}^{ab},
\end{aligned}
\end{equation}
where we used the relations
\begin{equation}
    \begin{aligned}
        -h^{ab} \square h_{ab} = & \,2(\bar{h}^{ab}-\frac{1}{2}\eta^{ab}\bar{h})\square (\bar{h}_{ab} - \frac{1}{2}\eta_{ab}\bar{h}) \\
        = & \,2\bar{h}^{ab} \square \bar{h}_{ab} - 2 \bar{h}\square \bar{h} + 2\bar{h}\square \bar{h} \\
        = & \,\bar{h}^{ab} \square \bar{h}_{ab},
    \end{aligned}
\end{equation}
\begin{equation}
        h \square h = -\bar{h}\square \bar{h},
\end{equation}
\begin{equation}
    \begin{aligned}
        - \partial_b h_{ac}\partial^c h^{ab} = & -\partial_b(\bar{h}_{ac} - \frac{1}{2}\eta_{ac}\bar{h})\partial^c (\bar{h}^{ab}-\frac{1}{2}\eta^{ab}\bar{h}) \\
        = & - \partial_b \bar{h}_{ac}\partial^c \bar{h}^{ab} + \partial_b \bar{h}\partial_a\bar{h}^{ab} - \frac{1}{4}\partial_b\bar{h} \partial^b\bar{h} \\
        = & - \partial_b \bar{h}_{ac}\partial^c \bar{h}^{ab} - \frac{1}{4}\partial_b \bar{h} \partial^b \bar{h},
    \end{aligned}
\end{equation}
\begin{equation}
    \begin{aligned}
        \frac{3}{2}\partial_c h_{ab}\partial^c h^{ab}
        = & \,\frac{3}{2}\partial_c (\bar{h}_{ab} - \frac{1}{2}\eta_{ab}\bar{h}) \partial^c (\bar{h}^{ab}-\frac{1}{2}\eta^{ab}\bar{h}) \\
        = & \,\frac{3}{2}\partial_c \bar{h}_{ab}\partial^c \bar{h}^{ab} -\frac{3}{2}\partial_c \bar{h} \partial^c \bar{h} + \frac{3}{2}\partial_ch\partial^c \bar{h} \\
        = & \,\frac{3}{2}\partial_c \bar{h}_{ab}\partial^c \bar{h}^{ab}.
    \end{aligned}
\end{equation}
Similarly the action for the GW is
\begin{equation}
\begin{aligned}
S = & \,\frac{1}{16\pi G} \int d^4x\left[  -\frac{1}{2} \bar{h}\square \bar{h} + \bar{h}^{ab}\square \bar{h}_{ab} - \partial_b \bar{h}_{ac}\partial^c \bar{h}^{ab} - \frac{1}{4}\partial_b\bar{h} \partial^b\bar{h} + \frac{3}{2}\partial_c \bar{h}_{ab}\partial^c \bar{h}^{ab} \right] \\
= & \,\frac{1}{16\pi } \int d^4x ~ \phi\left[  -\frac{1}{2} \bar{h}\square \bar{h} + \bar{h}^{ab}\square \bar{h}_{ab} - \partial_b \bar{h}_{ac}\partial^c \bar{h}^{ab} - \frac{1}{4}\partial_b\bar{h} \partial^b\bar{h} + \frac{3}{2}\partial_c \bar{h}_{ab}\partial^c \bar{h}^{ab} \right].
\end{aligned}
\end{equation}
We work in TT gauge,
\begin{equation}
\begin{aligned}
S = & \frac{1}{16\pi } \int d^4x ~ \phi\left[ \bar{h}^{ij}\square \bar{h}_{ij} - \partial_j \bar{h}_{ik}\partial^k \bar{h}^{ij} + \frac{3}{2}\partial_c \bar{h}_{ij}\partial^c \bar{h}^{ij} \right], \\
\end{aligned}
\end{equation}
and EOM becomes
\begin{equation}
\begin{aligned}
\phi\square \bar{h}_{ij} + \square (\phi \bar{h}_{ij}) + 2\partial^k(\phi\partial_j \bar{h}_{ik}) - \frac{3}{2}\partial^a (\phi \partial_a \bar{h}_{ij}) = 0,
\end{aligned}
\end{equation}
or
\begin{equation}
\begin{aligned}
\square \bar{h}_{ij} + \left(2\phi^{-1}\square \phi \right)\bar{h}_{ij} + \partial^a \ln\phi \partial_a\bar{h}_{ij} + 4\partial^k \ln\phi\partial_j \bar{h}_{ik} = 0.
\end{aligned}
\end{equation}
Assuming $\phi = \phi(\vec{r})$, we obtain the differential equations of $\bar{h}_{ij}$
\begin{equation}
\begin{aligned}
\square \bar{h}_{ij} + \left(\frac{2}{\phi}\nabla^2 \phi\right) \bar{h}_{ij} + \partial^k \ln\phi \partial_k \bar{h}_{ij} + 4\partial^k \ln\phi\partial_j \bar{h}_{ik} = 0.
\end{aligned}
\end{equation}
We then separate the variables as $\bar{h}_{ij} = H_{ij}e^{-i\omega t}$ so the spatial differential equations become
\begin{equation}
\begin{aligned}
\nabla^2H_{ij} + \left[ \omega^2 + \frac{2}{\phi}\nabla^2 \phi\right]H_{ij} + \partial^k \ln\phi \partial_k H_{ij} + 4\partial^k \ln\phi\partial_j H_{ik} = 0.
\end{aligned}
\end{equation}
Because the GW travels along $z$-axis, the $H_{ij}$ is the function of $z$ only,
\begin{equation}
\begin{aligned}
    \partial_z^2 H_{ij} + \left[ \omega^2 + \frac{2}{\phi}\partial_z^2 \phi \right]H_{ij} + \partial_z \ln\phi \partial_z H_{ij} + 4\delta^z_j\partial^k \ln\phi\partial_z H_{ik} = 0.
\end{aligned}
\end{equation}
In TT gauge, $H_{ij}$ furthermore only has $x,y$ components so the equations become
\begin{equation}\label{ehz}
\begin{aligned}
    \partial_z^2 h+ \partial_z \ln\phi\,\partial_z h + (\omega^2 + 2\phi^{-1}\partial_z^2 \phi )h  = 0,
\end{aligned}
\end{equation}
where $h$ represents $h_+$ or $h_\times$, and approximately this system has a frequency shift $\frac{\phi^{-1}\partial_z^2\phi}{\omega^2}\omega$.
In addition, if $\phi \neq \phi(\vec{r})$ but $\phi = \phi(t)$, the EOM becomes
\begin{equation}
\begin{aligned}
\frac{1}{2}\square \bar{h}_{ij} + \frac{\bar{h}_{ij}}{\phi}\square \phi + \frac{1}{2} \partial^a \ln\phi \partial_a\bar{h}_{ij} + 2\partial^k \ln\phi\partial_j \bar{h}_{ik} = 0,
\end{aligned}
\end{equation}
or after variable separation $\bar{h}_{ij} = C_{ij}e^{ikz}H(t)$,
\begin{equation}\label{eht}
\begin{aligned}
\partial_t^2H + \partial_t\ln\phi\,\partial_tH + (k^2 + 2\phi^{-1}\partial_t^2 \phi) H = 0,
\end{aligned}
\end{equation}
which has a similar form like eq.\eqref{ehz}.

Again, obviously eq.\eqref{ehz} has the form like eq.\eqref{dessoleq} and now we use definitions and methods in Appendix \ref{sec:appa}. For eq.\eqref{ehz},
\begin{equation}
    p=\frac{1}{2}\partial_z \ln\phi,\quad q=2\phi^{-1}\partial_z^2 \phi,
\end{equation}
therefore $\Gamma\propto\phi^{1/2}\propto G^{-1/2}$ and $\Xi=-\frac{1}{4}\phi^{-2}(\partial_z\phi)^2-\frac{3}{2}\phi^{-1}\partial_z^2\phi=-\frac{13}{4}G^{-2}(\partial_zG)^2+\frac{3}{2}G^{-1}\partial_z^2G$. If $G$ takes the Yukawa form eq.\eqref{Yukawa}, there will be
\begin{equation}
    \Xi=\frac{1}{\lambda^2}\frac{-7[\alpha(z/\lambda)e^{-z/\lambda}]^2-6[\alpha e^{-z/\lambda}]^2+6[\alpha(z/\lambda)e^{-z/\lambda}]-6[\alpha e^{-z/\lambda}]}{4[\alpha(z/\lambda+1)e^{-z/\lambda}+1]^2}.
\end{equation}
Again, if $\left\lvert\frac{ \Xi}{\omega^2}\right\rvert \ll 1$ and $1/\lambda\ll\omega/c$, which means that $G$ has minor correction and slow variation, the amplitude of GW will be approximately proportional to $G^{1/2}$ and the phase of GW will have an approximate modification $-\omega\int \frac{1}{2}\frac{\Xi}{\omega^2}\,d z$.

\appendix

\section{Solving Wave Equation}\label{sec:appa}

In this appendix, we solve the wave equation. By expressing $h_{ij}$ as a Fourier transform, the
wave equation for each Fourier amplitude becomes the following general linear form,
\begin{equation}
    \frac{d^2}{d z^2}H(z)+2p(z)\frac{d}{d z}H(z)+\left[\omega^2+q(z)\right]H(z)=0.\label{dessoleq}
\end{equation}

For any $H(z)$, there are real function $A(z)$ and $\Phi(z)$ which satisfy
\begin{equation}\label{HAphi}
    H=Ae^{i\Phi}.
\end{equation}
We plug eq.\eqref{HAphi} into eq.\eqref{dessoleq} and then divide both sides of the equation by $e^{i\Phi}$. From the real and imaginary part of the result, we achieve two equations,
\begin{equation}\label{rpart}
    \frac{d^2 A}{d z^2}+2p\frac{d A}{d z}+\left[\omega^2\left(1-\frac{k^2}{\omega^2}\right)+q\right]A=0,
\end{equation}
\begin{equation}\label{ipart}
    2\frac{d A}{d z}k+A\frac{d k}{d z}+2pAk=0,
\end{equation}
where $k=\frac{d \Phi}{d z}$. It is no matter for us to suppose that $A>0$ and $k>0$, and then
\begin{equation}
    2\frac{1}{A}\frac{d A}{d z}+\frac{1}{k}\frac{d k}{d z}+2p=0,
\end{equation}
therefore
\begin{equation}\label{Apk}
    A\propto e^{-\int p\,dz}k^{-1/2}.
\end{equation}
We plug eq.\eqref{Apk} into eq.\eqref{rpart} and then achieve
\begin{equation}\label{equK0}
    \frac{d^2 K}{d z^2}-\left(\frac{1}{\Gamma}\frac{d^2\Gamma}{d z^2}-q\right)K+\omega^2K(1-K^{-4})=0,
\end{equation}
where $\Gamma=e^{\int p \,dz}$ and $K=(k/\omega)^{-1/2}$. If $A=\check{A}$ and $k=+\check{k}$ satisfy eq.\eqref{rpart} and eq.\eqref{ipart}, $A=\check{A}$ and $k=-\check{k}$ will satisfy eq.\eqref{rpart} and eq.\eqref{ipart} as well. This means that since eq.\eqref{dessoleq} is linear, the general solution of eq.\eqref{dessoleq} can be expressed as
\begin{equation}\label{solution}
    H=C_+\Gamma^{-1}\check{K}e^{+i\omega\int  \check{K}^{-2}\,d z}+C_-\Gamma^{-1}\check{K}e^{-i\omega\int  \check{K}^{-2}\,d z},
\end{equation}
where $\check{K}$ is a particular solution of eq.\eqref{equK0}.

Now we need to find a particular solution of eq.\eqref{equK0}. We let $\Xi=\frac{1}{\Gamma}\frac{d^2\Gamma}{d z^2}-q$ and make $\omega=1$, which means that we are now using natural units where $c=1$ and $\omega=1$, then
\begin{equation}\label{equK}
    \frac{d^2 K}{d z^2}+K[(1-\Xi)-K^{-4}]=0.
\end{equation}
If $\Xi=\text{const}$, it is easy to find a particular solution
\begin{equation}\label{K0}
    K=(1-\Xi)^{-1/4}=1+\frac{1}{4}\Xi+\frac{5}{32}\Xi^2+O(\Xi^3),
\end{equation}
which means
\begin{equation}
    k=(1-\Xi)^{1/2}=1-\frac{1}{2}\Xi-\frac{1}{8}\Xi^2+O(\Xi^3),
\end{equation}
and this is similar to which of damped oscillations. If $\Xi\neq\text{const}$, we are interested in a special situation that $\Xi(z)=\kappa^2\tilde{\Xi}(\tilde{z})$, where $\tilde{z}=\kappa z$ and $\kappa$ is a small amount. The reason which leads to our interest is that if gravitational constant $G$ takes the Yukawa form, we can make parameter $\lambda$ in Yukawa form satisfies $\kappa=1/\lambda$ when $1/\lambda\ll\omega/c$. In this situation,
\begin{equation}\label{equKs}
    K^3\frac{d^2 K}{d \tilde{z}^2}\kappa^2-K^4\tilde{\Xi}(\tilde{z})\kappa^2+K^4-1=0.
\end{equation}
We suppose that
\begin{equation}\label{K}
    K=\sum_{n=0}^\infty K_n(\tilde{z})\kappa^{2n},
\end{equation}
and if we plug eq.\eqref{K} into eq.\eqref{equKs}, we will maintain a power series of $\kappa^2$ which is equal to $0$. We make each of the coefficients of this power series equal to $0$, for example,
\begin{gather}
    K_0^4-1=0,\\
    K_0^3K_0''-K_0^4\tilde{\Xi}+4K_0^3K_1=0,\\
    (K_0^3K_1''+3K_0^2K_1K_0'')-4K_0^3K_1\tilde{\Xi}+(4K_0^3K_2+6K_0^2K_1^2)=0.
\end{gather}
Finally, by solving all of these equations in order, we will find each of the coefficients in eq.\eqref{K}, for example,
\begin{gather}
    K_0=1,\\
    K_1=\frac{1}{4}\tilde{\Xi},\\
    K_2=\frac{5}{32}\tilde{\Xi}^2-\frac{1}{16}\frac{d^2\tilde{\Xi}}{d\tilde{z}^2},
\end{gather}
and these coefficients will make our particular solution of $K$ similar to eq.\eqref{K0}.

\section{Possibility of Measurement}

Firstly, if gravitational constant $G$ has a spatial variation, then according to eq.\eqref{solution}, the GW do have a amplitude correction $\Gamma^{-1}(z){K}(z)$. However, the detector can only measure GW amplitudes in where the detector is, therefore it is impossible for the detector to compare different GW amplitudes in different places. Also, according to eq.\eqref{solution}, GWs of different frequencies have the same amplitude correction $\Gamma^{-1}(z){K}(z)$, therefore it is impossible for the detector to compare different amplitudes of GWs of different frequencies to measure the spatial variation of $G$.

Secondly, according to eq.\eqref{solution}, GWs of different frequencies propagate in the same speed $c/{K}(z)^{-2}$, where $c$ is the speed of light in vacuum, which means phase differences between GWs of different frequencies will keep unchanged during GW propagations, therefore it is impossible for the detector to measure phase differences between GWs of different frequencies to measure the spatial variation of $G$.

In conclusion, it is impossible for us to measure the spatial variation of $G$ during GW propagations by only GW observations.

\bibliography{reference}

\end{document}
